
% Default to the notebook output style

    


% Inherit from the specified cell style.




    
\documentclass[11pt]{article}

    
    
    \usepackage[T1]{fontenc}
    % Nicer default font (+ math font) than Computer Modern for most use cases
    \usepackage{mathpazo}

    % Basic figure setup, for now with no caption control since it's done
    % automatically by Pandoc (which extracts ![](path) syntax from Markdown).
    \usepackage{graphicx}
    % We will generate all images so they have a width \maxwidth. This means
    % that they will get their normal width if they fit onto the page, but
    % are scaled down if they would overflow the margins.
    \makeatletter
    \def\maxwidth{\ifdim\Gin@nat@width>\linewidth\linewidth
    \else\Gin@nat@width\fi}
    \makeatother
    \let\Oldincludegraphics\includegraphics
    % Set max figure width to be 80% of text width, for now hardcoded.
    \renewcommand{\includegraphics}[1]{\Oldincludegraphics[width=.8\maxwidth]{#1}}
    % Ensure that by default, figures have no caption (until we provide a
    % proper Figure object with a Caption API and a way to capture that
    % in the conversion process - todo).
    \usepackage{caption}
    \DeclareCaptionLabelFormat{nolabel}{}
    \captionsetup{labelformat=nolabel}

    \usepackage{adjustbox} % Used to constrain images to a maximum size 
    \usepackage{xcolor} % Allow colors to be defined
    \usepackage{enumerate} % Needed for markdown enumerations to work
    \usepackage{geometry} % Used to adjust the document margins
    \usepackage{amsmath} % Equations
    \usepackage{amssymb} % Equations
    \usepackage{textcomp} % defines textquotesingle
    % Hack from http://tex.stackexchange.com/a/47451/13684:
    \AtBeginDocument{%
        \def\PYZsq{\textquotesingle}% Upright quotes in Pygmentized code
    }
    \usepackage{upquote} % Upright quotes for verbatim code
    \usepackage{eurosym} % defines \euro
    \usepackage[mathletters]{ucs} % Extended unicode (utf-8) support
    \usepackage[utf8x]{inputenc} % Allow utf-8 characters in the tex document
    \usepackage{fancyvrb} % verbatim replacement that allows latex
    \usepackage{grffile} % extends the file name processing of package graphics 
                         % to support a larger range 
    % The hyperref package gives us a pdf with properly built
    % internal navigation ('pdf bookmarks' for the table of contents,
    % internal cross-reference links, web links for URLs, etc.)
    \usepackage{hyperref}
    \usepackage{longtable} % longtable support required by pandoc >1.10
    \usepackage{booktabs}  % table support for pandoc > 1.12.2
    \usepackage[inline]{enumitem} % IRkernel/repr support (it uses the enumerate* environment)
    \usepackage[normalem]{ulem} % ulem is needed to support strikethroughs (\sout)
                                % normalem makes italics be italics, not underlines
    

    
    
    % Colors for the hyperref package
    \definecolor{urlcolor}{rgb}{0,.145,.698}
    \definecolor{linkcolor}{rgb}{.71,0.21,0.01}
    \definecolor{citecolor}{rgb}{.12,.54,.11}

    % ANSI colors
    \definecolor{ansi-black}{HTML}{3E424D}
    \definecolor{ansi-black-intense}{HTML}{282C36}
    \definecolor{ansi-red}{HTML}{E75C58}
    \definecolor{ansi-red-intense}{HTML}{B22B31}
    \definecolor{ansi-green}{HTML}{00A250}
    \definecolor{ansi-green-intense}{HTML}{007427}
    \definecolor{ansi-yellow}{HTML}{DDB62B}
    \definecolor{ansi-yellow-intense}{HTML}{B27D12}
    \definecolor{ansi-blue}{HTML}{208FFB}
    \definecolor{ansi-blue-intense}{HTML}{0065CA}
    \definecolor{ansi-magenta}{HTML}{D160C4}
    \definecolor{ansi-magenta-intense}{HTML}{A03196}
    \definecolor{ansi-cyan}{HTML}{60C6C8}
    \definecolor{ansi-cyan-intense}{HTML}{258F8F}
    \definecolor{ansi-white}{HTML}{C5C1B4}
    \definecolor{ansi-white-intense}{HTML}{A1A6B2}

    % commands and environments needed by pandoc snippets
    % extracted from the output of `pandoc -s`
    \providecommand{\tightlist}{%
      \setlength{\itemsep}{0pt}\setlength{\parskip}{0pt}}
    \DefineVerbatimEnvironment{Highlighting}{Verbatim}{commandchars=\\\{\}}
    % Add ',fontsize=\small' for more characters per line
    \newenvironment{Shaded}{}{}
    \newcommand{\KeywordTok}[1]{\textcolor[rgb]{0.00,0.44,0.13}{\textbf{{#1}}}}
    \newcommand{\DataTypeTok}[1]{\textcolor[rgb]{0.56,0.13,0.00}{{#1}}}
    \newcommand{\DecValTok}[1]{\textcolor[rgb]{0.25,0.63,0.44}{{#1}}}
    \newcommand{\BaseNTok}[1]{\textcolor[rgb]{0.25,0.63,0.44}{{#1}}}
    \newcommand{\FloatTok}[1]{\textcolor[rgb]{0.25,0.63,0.44}{{#1}}}
    \newcommand{\CharTok}[1]{\textcolor[rgb]{0.25,0.44,0.63}{{#1}}}
    \newcommand{\StringTok}[1]{\textcolor[rgb]{0.25,0.44,0.63}{{#1}}}
    \newcommand{\CommentTok}[1]{\textcolor[rgb]{0.38,0.63,0.69}{\textit{{#1}}}}
    \newcommand{\OtherTok}[1]{\textcolor[rgb]{0.00,0.44,0.13}{{#1}}}
    \newcommand{\AlertTok}[1]{\textcolor[rgb]{1.00,0.00,0.00}{\textbf{{#1}}}}
    \newcommand{\FunctionTok}[1]{\textcolor[rgb]{0.02,0.16,0.49}{{#1}}}
    \newcommand{\RegionMarkerTok}[1]{{#1}}
    \newcommand{\ErrorTok}[1]{\textcolor[rgb]{1.00,0.00,0.00}{\textbf{{#1}}}}
    \newcommand{\NormalTok}[1]{{#1}}
    
    % Additional commands for more recent versions of Pandoc
    \newcommand{\ConstantTok}[1]{\textcolor[rgb]{0.53,0.00,0.00}{{#1}}}
    \newcommand{\SpecialCharTok}[1]{\textcolor[rgb]{0.25,0.44,0.63}{{#1}}}
    \newcommand{\VerbatimStringTok}[1]{\textcolor[rgb]{0.25,0.44,0.63}{{#1}}}
    \newcommand{\SpecialStringTok}[1]{\textcolor[rgb]{0.73,0.40,0.53}{{#1}}}
    \newcommand{\ImportTok}[1]{{#1}}
    \newcommand{\DocumentationTok}[1]{\textcolor[rgb]{0.73,0.13,0.13}{\textit{{#1}}}}
    \newcommand{\AnnotationTok}[1]{\textcolor[rgb]{0.38,0.63,0.69}{\textbf{\textit{{#1}}}}}
    \newcommand{\CommentVarTok}[1]{\textcolor[rgb]{0.38,0.63,0.69}{\textbf{\textit{{#1}}}}}
    \newcommand{\VariableTok}[1]{\textcolor[rgb]{0.10,0.09,0.49}{{#1}}}
    \newcommand{\ControlFlowTok}[1]{\textcolor[rgb]{0.00,0.44,0.13}{\textbf{{#1}}}}
    \newcommand{\OperatorTok}[1]{\textcolor[rgb]{0.40,0.40,0.40}{{#1}}}
    \newcommand{\BuiltInTok}[1]{{#1}}
    \newcommand{\ExtensionTok}[1]{{#1}}
    \newcommand{\PreprocessorTok}[1]{\textcolor[rgb]{0.74,0.48,0.00}{{#1}}}
    \newcommand{\AttributeTok}[1]{\textcolor[rgb]{0.49,0.56,0.16}{{#1}}}
    \newcommand{\InformationTok}[1]{\textcolor[rgb]{0.38,0.63,0.69}{\textbf{\textit{{#1}}}}}
    \newcommand{\WarningTok}[1]{\textcolor[rgb]{0.38,0.63,0.69}{\textbf{\textit{{#1}}}}}
    
    
    % Define a nice break command that doesn't care if a line doesn't already
    % exist.
    \def\br{\hspace*{\fill} \\* }
    % Math Jax compatability definitions
    \def\gt{>}
    \def\lt{<}
    % Document parameters
    \title{regression-assignment-02-Copy2}
    
    
    

    % Pygments definitions
    
\makeatletter
\def\PY@reset{\let\PY@it=\relax \let\PY@bf=\relax%
    \let\PY@ul=\relax \let\PY@tc=\relax%
    \let\PY@bc=\relax \let\PY@ff=\relax}
\def\PY@tok#1{\csname PY@tok@#1\endcsname}
\def\PY@toks#1+{\ifx\relax#1\empty\else%
    \PY@tok{#1}\expandafter\PY@toks\fi}
\def\PY@do#1{\PY@bc{\PY@tc{\PY@ul{%
    \PY@it{\PY@bf{\PY@ff{#1}}}}}}}
\def\PY#1#2{\PY@reset\PY@toks#1+\relax+\PY@do{#2}}

\expandafter\def\csname PY@tok@cs\endcsname{\let\PY@it=\textit\def\PY@tc##1{\textcolor[rgb]{0.25,0.50,0.50}{##1}}}
\expandafter\def\csname PY@tok@c\endcsname{\let\PY@it=\textit\def\PY@tc##1{\textcolor[rgb]{0.25,0.50,0.50}{##1}}}
\expandafter\def\csname PY@tok@sc\endcsname{\def\PY@tc##1{\textcolor[rgb]{0.73,0.13,0.13}{##1}}}
\expandafter\def\csname PY@tok@sb\endcsname{\def\PY@tc##1{\textcolor[rgb]{0.73,0.13,0.13}{##1}}}
\expandafter\def\csname PY@tok@kd\endcsname{\let\PY@bf=\textbf\def\PY@tc##1{\textcolor[rgb]{0.00,0.50,0.00}{##1}}}
\expandafter\def\csname PY@tok@ow\endcsname{\let\PY@bf=\textbf\def\PY@tc##1{\textcolor[rgb]{0.67,0.13,1.00}{##1}}}
\expandafter\def\csname PY@tok@dl\endcsname{\def\PY@tc##1{\textcolor[rgb]{0.73,0.13,0.13}{##1}}}
\expandafter\def\csname PY@tok@gu\endcsname{\let\PY@bf=\textbf\def\PY@tc##1{\textcolor[rgb]{0.50,0.00,0.50}{##1}}}
\expandafter\def\csname PY@tok@o\endcsname{\def\PY@tc##1{\textcolor[rgb]{0.40,0.40,0.40}{##1}}}
\expandafter\def\csname PY@tok@s2\endcsname{\def\PY@tc##1{\textcolor[rgb]{0.73,0.13,0.13}{##1}}}
\expandafter\def\csname PY@tok@se\endcsname{\let\PY@bf=\textbf\def\PY@tc##1{\textcolor[rgb]{0.73,0.40,0.13}{##1}}}
\expandafter\def\csname PY@tok@nc\endcsname{\let\PY@bf=\textbf\def\PY@tc##1{\textcolor[rgb]{0.00,0.00,1.00}{##1}}}
\expandafter\def\csname PY@tok@gp\endcsname{\let\PY@bf=\textbf\def\PY@tc##1{\textcolor[rgb]{0.00,0.00,0.50}{##1}}}
\expandafter\def\csname PY@tok@no\endcsname{\def\PY@tc##1{\textcolor[rgb]{0.53,0.00,0.00}{##1}}}
\expandafter\def\csname PY@tok@fm\endcsname{\def\PY@tc##1{\textcolor[rgb]{0.00,0.00,1.00}{##1}}}
\expandafter\def\csname PY@tok@kp\endcsname{\def\PY@tc##1{\textcolor[rgb]{0.00,0.50,0.00}{##1}}}
\expandafter\def\csname PY@tok@ne\endcsname{\let\PY@bf=\textbf\def\PY@tc##1{\textcolor[rgb]{0.82,0.25,0.23}{##1}}}
\expandafter\def\csname PY@tok@mh\endcsname{\def\PY@tc##1{\textcolor[rgb]{0.40,0.40,0.40}{##1}}}
\expandafter\def\csname PY@tok@nl\endcsname{\def\PY@tc##1{\textcolor[rgb]{0.63,0.63,0.00}{##1}}}
\expandafter\def\csname PY@tok@nd\endcsname{\def\PY@tc##1{\textcolor[rgb]{0.67,0.13,1.00}{##1}}}
\expandafter\def\csname PY@tok@cpf\endcsname{\let\PY@it=\textit\def\PY@tc##1{\textcolor[rgb]{0.25,0.50,0.50}{##1}}}
\expandafter\def\csname PY@tok@mb\endcsname{\def\PY@tc##1{\textcolor[rgb]{0.40,0.40,0.40}{##1}}}
\expandafter\def\csname PY@tok@gi\endcsname{\def\PY@tc##1{\textcolor[rgb]{0.00,0.63,0.00}{##1}}}
\expandafter\def\csname PY@tok@kt\endcsname{\def\PY@tc##1{\textcolor[rgb]{0.69,0.00,0.25}{##1}}}
\expandafter\def\csname PY@tok@vm\endcsname{\def\PY@tc##1{\textcolor[rgb]{0.10,0.09,0.49}{##1}}}
\expandafter\def\csname PY@tok@err\endcsname{\def\PY@bc##1{\setlength{\fboxsep}{0pt}\fcolorbox[rgb]{1.00,0.00,0.00}{1,1,1}{\strut ##1}}}
\expandafter\def\csname PY@tok@ss\endcsname{\def\PY@tc##1{\textcolor[rgb]{0.10,0.09,0.49}{##1}}}
\expandafter\def\csname PY@tok@gs\endcsname{\let\PY@bf=\textbf}
\expandafter\def\csname PY@tok@c1\endcsname{\let\PY@it=\textit\def\PY@tc##1{\textcolor[rgb]{0.25,0.50,0.50}{##1}}}
\expandafter\def\csname PY@tok@sr\endcsname{\def\PY@tc##1{\textcolor[rgb]{0.73,0.40,0.53}{##1}}}
\expandafter\def\csname PY@tok@si\endcsname{\let\PY@bf=\textbf\def\PY@tc##1{\textcolor[rgb]{0.73,0.40,0.53}{##1}}}
\expandafter\def\csname PY@tok@sa\endcsname{\def\PY@tc##1{\textcolor[rgb]{0.73,0.13,0.13}{##1}}}
\expandafter\def\csname PY@tok@gh\endcsname{\let\PY@bf=\textbf\def\PY@tc##1{\textcolor[rgb]{0.00,0.00,0.50}{##1}}}
\expandafter\def\csname PY@tok@il\endcsname{\def\PY@tc##1{\textcolor[rgb]{0.40,0.40,0.40}{##1}}}
\expandafter\def\csname PY@tok@nn\endcsname{\let\PY@bf=\textbf\def\PY@tc##1{\textcolor[rgb]{0.00,0.00,1.00}{##1}}}
\expandafter\def\csname PY@tok@go\endcsname{\def\PY@tc##1{\textcolor[rgb]{0.53,0.53,0.53}{##1}}}
\expandafter\def\csname PY@tok@kc\endcsname{\let\PY@bf=\textbf\def\PY@tc##1{\textcolor[rgb]{0.00,0.50,0.00}{##1}}}
\expandafter\def\csname PY@tok@s1\endcsname{\def\PY@tc##1{\textcolor[rgb]{0.73,0.13,0.13}{##1}}}
\expandafter\def\csname PY@tok@mi\endcsname{\def\PY@tc##1{\textcolor[rgb]{0.40,0.40,0.40}{##1}}}
\expandafter\def\csname PY@tok@ch\endcsname{\let\PY@it=\textit\def\PY@tc##1{\textcolor[rgb]{0.25,0.50,0.50}{##1}}}
\expandafter\def\csname PY@tok@gr\endcsname{\def\PY@tc##1{\textcolor[rgb]{1.00,0.00,0.00}{##1}}}
\expandafter\def\csname PY@tok@bp\endcsname{\def\PY@tc##1{\textcolor[rgb]{0.00,0.50,0.00}{##1}}}
\expandafter\def\csname PY@tok@m\endcsname{\def\PY@tc##1{\textcolor[rgb]{0.40,0.40,0.40}{##1}}}
\expandafter\def\csname PY@tok@cp\endcsname{\def\PY@tc##1{\textcolor[rgb]{0.74,0.48,0.00}{##1}}}
\expandafter\def\csname PY@tok@na\endcsname{\def\PY@tc##1{\textcolor[rgb]{0.49,0.56,0.16}{##1}}}
\expandafter\def\csname PY@tok@sd\endcsname{\let\PY@it=\textit\def\PY@tc##1{\textcolor[rgb]{0.73,0.13,0.13}{##1}}}
\expandafter\def\csname PY@tok@vc\endcsname{\def\PY@tc##1{\textcolor[rgb]{0.10,0.09,0.49}{##1}}}
\expandafter\def\csname PY@tok@sh\endcsname{\def\PY@tc##1{\textcolor[rgb]{0.73,0.13,0.13}{##1}}}
\expandafter\def\csname PY@tok@kn\endcsname{\let\PY@bf=\textbf\def\PY@tc##1{\textcolor[rgb]{0.00,0.50,0.00}{##1}}}
\expandafter\def\csname PY@tok@kr\endcsname{\let\PY@bf=\textbf\def\PY@tc##1{\textcolor[rgb]{0.00,0.50,0.00}{##1}}}
\expandafter\def\csname PY@tok@mo\endcsname{\def\PY@tc##1{\textcolor[rgb]{0.40,0.40,0.40}{##1}}}
\expandafter\def\csname PY@tok@vi\endcsname{\def\PY@tc##1{\textcolor[rgb]{0.10,0.09,0.49}{##1}}}
\expandafter\def\csname PY@tok@vg\endcsname{\def\PY@tc##1{\textcolor[rgb]{0.10,0.09,0.49}{##1}}}
\expandafter\def\csname PY@tok@mf\endcsname{\def\PY@tc##1{\textcolor[rgb]{0.40,0.40,0.40}{##1}}}
\expandafter\def\csname PY@tok@gd\endcsname{\def\PY@tc##1{\textcolor[rgb]{0.63,0.00,0.00}{##1}}}
\expandafter\def\csname PY@tok@nb\endcsname{\def\PY@tc##1{\textcolor[rgb]{0.00,0.50,0.00}{##1}}}
\expandafter\def\csname PY@tok@nf\endcsname{\def\PY@tc##1{\textcolor[rgb]{0.00,0.00,1.00}{##1}}}
\expandafter\def\csname PY@tok@nv\endcsname{\def\PY@tc##1{\textcolor[rgb]{0.10,0.09,0.49}{##1}}}
\expandafter\def\csname PY@tok@ni\endcsname{\let\PY@bf=\textbf\def\PY@tc##1{\textcolor[rgb]{0.60,0.60,0.60}{##1}}}
\expandafter\def\csname PY@tok@s\endcsname{\def\PY@tc##1{\textcolor[rgb]{0.73,0.13,0.13}{##1}}}
\expandafter\def\csname PY@tok@ge\endcsname{\let\PY@it=\textit}
\expandafter\def\csname PY@tok@nt\endcsname{\let\PY@bf=\textbf\def\PY@tc##1{\textcolor[rgb]{0.00,0.50,0.00}{##1}}}
\expandafter\def\csname PY@tok@sx\endcsname{\def\PY@tc##1{\textcolor[rgb]{0.00,0.50,0.00}{##1}}}
\expandafter\def\csname PY@tok@k\endcsname{\let\PY@bf=\textbf\def\PY@tc##1{\textcolor[rgb]{0.00,0.50,0.00}{##1}}}
\expandafter\def\csname PY@tok@w\endcsname{\def\PY@tc##1{\textcolor[rgb]{0.73,0.73,0.73}{##1}}}
\expandafter\def\csname PY@tok@cm\endcsname{\let\PY@it=\textit\def\PY@tc##1{\textcolor[rgb]{0.25,0.50,0.50}{##1}}}
\expandafter\def\csname PY@tok@gt\endcsname{\def\PY@tc##1{\textcolor[rgb]{0.00,0.27,0.87}{##1}}}

\def\PYZbs{\char`\\}
\def\PYZus{\char`\_}
\def\PYZob{\char`\{}
\def\PYZcb{\char`\}}
\def\PYZca{\char`\^}
\def\PYZam{\char`\&}
\def\PYZlt{\char`\<}
\def\PYZgt{\char`\>}
\def\PYZsh{\char`\#}
\def\PYZpc{\char`\%}
\def\PYZdl{\char`\$}
\def\PYZhy{\char`\-}
\def\PYZsq{\char`\'}
\def\PYZdq{\char`\"}
\def\PYZti{\char`\~}
% for compatibility with earlier versions
\def\PYZat{@}
\def\PYZlb{[}
\def\PYZrb{]}
\makeatother


    % Exact colors from NB
    \definecolor{incolor}{rgb}{0.0, 0.0, 0.5}
    \definecolor{outcolor}{rgb}{0.545, 0.0, 0.0}



    
    % Prevent overflowing lines due to hard-to-break entities
    \sloppy 
    % Setup hyperref package
    \hypersetup{
      breaklinks=true,  % so long urls are correctly broken across lines
      colorlinks=true,
      urlcolor=urlcolor,
      linkcolor=linkcolor,
      citecolor=citecolor,
      }
    % Slightly bigger margins than the latex defaults
    
    \geometry{verbose,tmargin=1in,bmargin=1in,lmargin=1in,rmargin=1in}
    
    

    \begin{document}
    
    
    \maketitle
    
    

    
    \section{INF283 \textbar{} Weekly Exercise 02 \textbar{}
Regression}\label{inf283-weekly-exercise-02-regression}

    \subsection{Table of Contents:}\label{table-of-contents}

Section \ref{univariate} * Section \ref{exercise1_1}

\begin{itemize}
\item
  Section \ref{exercise1_2}
\item
  Section \ref{exercise1_3}
\item
  Section \ref{exercise1_4}
\item
  Section \ref{exercise1_5}
\end{itemize}

Section \ref{multivariate}

\begin{itemize}
\item
  Section \ref{exercise2_1}
\item
  Section \ref{exercise2_2}
\item
  Section \ref{exercise2_3}
\end{itemize}

Section \ref{logistic}

\begin{itemize}
\tightlist
\item
  Section \ref{exercise3_1}
\end{itemize}

    \section{1. Univariate Linear Regression
}\label{univariate-linear-regression}

    \section{Linear Regression with Gradient
Descent}\label{linear-regression-with-gradient-descent}

    \begin{Verbatim}[commandchars=\\\{\}]
{\color{incolor}In [{\color{incolor}1}]:} \PY{k+kn}{import} \PY{n+nn}{numpy} \PY{k}{as} \PY{n+nn}{np}
        \PY{k+kn}{import} \PY{n+nn}{matplotlib}\PY{n+nn}{.}\PY{n+nn}{pyplot} \PY{k}{as} \PY{n+nn}{plt}
        \PY{o}{\PYZpc{}}\PY{k}{matplotlib} inline
        \PY{k+kn}{import} \PY{n+nn}{pandas} \PY{k}{as} \PY{n+nn}{pd}
\end{Verbatim}


    \subsubsection{Finding the mean squared error
(MSE)}\label{finding-the-mean-squared-error-mse}

    \begin{Verbatim}[commandchars=\\\{\}]
{\color{incolor}In [{\color{incolor}2}]:} \PY{c+c1}{\PYZsh{} We are now going to define a function which takes the parameters (w0 and w1) of }
        \PY{c+c1}{\PYZsh{} a line and then finds the mean\PYZhy{}squared error between the user\PYZhy{}specified points }
        \PY{c+c1}{\PYZsh{} and the line. }
        \PY{k}{def} \PY{n+nf}{compute\PYZus{}error\PYZus{}for\PYZus{}line\PYZus{}given\PYZus{}points}\PY{p}{(}\PY{n}{w0}\PY{p}{,} \PY{n}{w1}\PY{p}{,} \PY{n}{points}\PY{p}{)}\PY{p}{:}
            \PY{n}{totalError} \PY{o}{=} \PY{l+m+mi}{0}
            \PY{k}{for} \PY{n}{i} \PY{o+ow}{in} \PY{n+nb}{range}\PY{p}{(}\PY{l+m+mi}{0}\PY{p}{,} \PY{n+nb}{len}\PY{p}{(}\PY{n}{points}\PY{p}{)}\PY{p}{)}\PY{p}{:}
                \PY{n}{x} \PY{o}{=} \PY{n}{points}\PY{p}{[}\PY{n}{i}\PY{p}{,} \PY{l+m+mi}{0}\PY{p}{]}
                \PY{n}{y} \PY{o}{=} \PY{n}{points}\PY{p}{[}\PY{n}{i}\PY{p}{,} \PY{l+m+mi}{1}\PY{p}{]}
                \PY{c+c1}{\PYZsh{} accumulate \PYZsq{}sum of square\PYZsq{} errors in totalError variable}
                \PY{n}{totalError} \PY{o}{+}\PY{o}{=} \PY{p}{(}\PY{n}{y} \PY{o}{\PYZhy{}} \PY{p}{(}\PY{n}{w1} \PY{o}{*} \PY{n}{x} \PY{o}{+} \PY{n}{w0}\PY{p}{)}\PY{p}{)} \PY{o}{*}\PY{o}{*} \PY{l+m+mi}{2}
            \PY{c+c1}{\PYZsh{} find mean of sum of squared errors    }
            \PY{n}{mse} \PY{o}{=} \PY{n}{totalError}\PY{o}{/}\PY{n+nb}{len}\PY{p}{(}\PY{n}{points}\PY{p}{)}
            \PY{k}{return} \PY{n}{mse}
        
        \PY{c+c1}{\PYZsh{} N.B.: Students who wish to do this excercise in R should implement this function in R themselves}
\end{Verbatim}


    \subsubsection{Gradient Descent}\label{gradient-descent}

    \begin{Verbatim}[commandchars=\\\{\}]
{\color{incolor}In [{\color{incolor}3}]:} \PY{k}{def} \PY{n+nf}{step\PYZus{}gradient}\PY{p}{(}\PY{n}{w0\PYZus{}current}\PY{p}{,} \PY{n}{w1\PYZus{}current}\PY{p}{,} \PY{n}{points}\PY{p}{,} \PY{n}{learningRate}\PY{p}{)}\PY{p}{:}
            \PY{c+c1}{\PYZsh{}initialize the partial derivatives for the cummlative sum}
            \PY{n}{w0\PYZus{}par\PYZus{}der} \PY{o}{=} \PY{l+m+mi}{0}
            \PY{n}{w1\PYZus{}par\PYZus{}der} \PY{o}{=} \PY{l+m+mi}{0}
            
            \PY{n}{n} \PY{o}{=} \PY{n+nb}{len}\PY{p}{(}\PY{n}{points}\PY{p}{)}
            
            \PY{c+c1}{\PYZsh{} computation for the summation}
            \PY{k}{for} \PY{n}{i} \PY{o+ow}{in} \PY{n+nb}{range}\PY{p}{(}\PY{l+m+mi}{0}\PY{p}{,} \PY{n+nb}{len}\PY{p}{(}\PY{n}{points}\PY{p}{)}\PY{p}{)}\PY{p}{:}
                \PY{n}{x} \PY{o}{=} \PY{n}{points}\PY{p}{[}\PY{n}{i}\PY{p}{,} \PY{l+m+mi}{0}\PY{p}{]}
                \PY{n}{y} \PY{o}{=} \PY{n}{points}\PY{p}{[}\PY{n}{i}\PY{p}{,} \PY{l+m+mi}{1}\PY{p}{]}
                \PY{c+c1}{\PYZsh{} partial derivative (of MSE) with respect to w0}
                \PY{n}{w0\PYZus{}par\PYZus{}der} \PY{o}{+}\PY{o}{=} \PY{p}{(}\PY{n}{y} \PY{o}{\PYZhy{}} \PY{p}{(}\PY{p}{(}\PY{n}{w1\PYZus{}current} \PY{o}{*} \PY{n}{x}\PY{p}{)} \PY{o}{+} \PY{n}{w0\PYZus{}current}\PY{p}{)}\PY{p}{)}
                \PY{c+c1}{\PYZsh{} partial derivative (of MSE) with respect to w1}
                \PY{n}{w1\PYZus{}par\PYZus{}der} \PY{o}{+}\PY{o}{=} \PY{n}{x} \PY{o}{*} \PY{p}{(}\PY{n}{y} \PY{o}{\PYZhy{}} \PY{p}{(}\PY{p}{(}\PY{n}{w1\PYZus{}current} \PY{o}{*} \PY{n}{x}\PY{p}{)} \PY{o}{+} \PY{n}{w0\PYZus{}current}\PY{p}{)}\PY{p}{)}
                
            \PY{c+c1}{\PYZsh{} multiplcation of summation results with \PYZhy{}2/n}
            \PY{n}{w0\PYZus{}par\PYZus{}der} \PY{o}{=} \PY{o}{\PYZhy{}}\PY{p}{(}\PY{l+m+mi}{2}\PY{o}{/}\PY{n}{n}\PY{p}{)} \PY{o}{*} \PY{n}{w0\PYZus{}par\PYZus{}der}
                \PY{c+c1}{\PYZsh{} partial derivative (of MSE) with respect to w1}
            \PY{n}{w1\PYZus{}par\PYZus{}der} \PY{o}{=} \PY{o}{\PYZhy{}}\PY{p}{(}\PY{l+m+mi}{2}\PY{o}{/}\PY{n}{n}\PY{p}{)} \PY{o}{*} \PY{n}{w1\PYZus{}par\PYZus{}der}
                 
            \PY{c+c1}{\PYZsh{} make a gradient vector from the partial derivatives    }
            \PY{n}{gradient\PYZus{}mse} \PY{o}{=} \PY{n}{np}\PY{o}{.}\PY{n}{array}\PY{p}{(}\PY{p}{[}\PY{n}{w0\PYZus{}par\PYZus{}der}\PY{p}{,} \PY{n}{w1\PYZus{}par\PYZus{}der}\PY{p}{]}\PY{p}{)}
            
            \PY{c+c1}{\PYZsh{} make a vector of weights}
            \PY{n}{weight\PYZus{}vector} \PY{o}{=} \PY{n}{np}\PY{o}{.}\PY{n}{array}\PY{p}{(}\PY{p}{[}\PY{n}{w0\PYZus{}current}\PY{p}{,} \PY{n}{w1\PYZus{}current}\PY{p}{]}\PY{p}{)}
            
            \PY{c+c1}{\PYZsh{} update rule for weights}
            \PY{n}{updated\PYZus{}weight\PYZus{}vector} \PY{o}{=} \PY{n}{weight\PYZus{}vector} \PY{o}{\PYZhy{}} \PY{p}{(}\PY{n}{learningRate} \PY{o}{*} \PY{n}{gradient\PYZus{}mse}\PY{p}{)}
            
            \PY{c+c1}{\PYZsh{} return the updated weight vector as a list}
            \PY{k}{return} \PY{n}{np}\PY{o}{.}\PY{n}{ndarray}\PY{o}{.}\PY{n}{tolist}\PY{p}{(}\PY{n}{updated\PYZus{}weight\PYZus{}vector}\PY{p}{)}
        
        \PY{c+c1}{\PYZsh{} N.B. Students who wish to do this excercise in R should implement this function in R themselves}
\end{Verbatim}


    \subsubsection{Running Gradient Descent
Iteratively}\label{running-gradient-descent-iteratively}

    \begin{Verbatim}[commandchars=\\\{\}]
{\color{incolor}In [{\color{incolor}4}]:} \PY{k}{def} \PY{n+nf}{gradient\PYZus{}descent\PYZus{}runner}\PY{p}{(}\PY{n}{points}\PY{p}{,} \PY{n}{starting\PYZus{}w0}\PY{p}{,} \PY{n}{starting\PYZus{}w1}\PY{p}{,} \PY{n}{learning\PYZus{}rate}\PY{p}{,} \PY{n}{num\PYZus{}iterations}\PY{p}{)}\PY{p}{:}
            \PY{n}{w0} \PY{o}{=} \PY{n}{starting\PYZus{}w0}
            \PY{n}{w1} \PY{o}{=} \PY{n}{starting\PYZus{}w1}
            \PY{k}{for} \PY{n}{i} \PY{o+ow}{in} \PY{n+nb}{range}\PY{p}{(}\PY{n}{num\PYZus{}iterations}\PY{p}{)}\PY{p}{:}
                \PY{n}{w0}\PY{p}{,} \PY{n}{w1} \PY{o}{=} \PY{n}{step\PYZus{}gradient}\PY{p}{(}\PY{n}{w0}\PY{p}{,} \PY{n}{w1}\PY{p}{,} \PY{n}{points}\PY{p}{,} \PY{n}{learning\PYZus{}rate}\PY{p}{)}
                \PY{n}{mse} \PY{o}{=} \PY{n}{compute\PYZus{}error\PYZus{}for\PYZus{}line\PYZus{}given\PYZus{}points}\PY{p}{(}\PY{n}{w0}\PY{p}{,} \PY{n}{w1}\PY{p}{,} \PY{n}{points}\PY{p}{)}
                \PY{n+nb}{print}\PY{p}{(}\PY{n}{f}\PY{l+s+s1}{\PYZsq{}}\PY{l+s+s1}{Iteration }\PY{l+s+s1}{\PYZob{}}\PY{l+s+s1}{i+1\PYZcb{}: w0=}\PY{l+s+si}{\PYZob{}w0:0.5f\PYZcb{}}\PY{l+s+s1}{, w1=}\PY{l+s+si}{\PYZob{}w1:0.5f\PYZcb{}}\PY{l+s+s1}{, mse=}\PY{l+s+si}{\PYZob{}mse:0.5f\PYZcb{}}\PY{l+s+s1}{\PYZsq{}}\PY{p}{)}
            \PY{k}{return} \PY{p}{[}\PY{n}{w0}\PY{p}{,} \PY{n}{w1}\PY{p}{,} \PY{n}{mse}\PY{p}{]}
        \PY{c+c1}{\PYZsh{} N.B.: Students who wish to do this exercise in R should implement this function in R themselves}
\end{Verbatim}


    \subsubsection{Bringing it all together}\label{bringing-it-all-together}

    \begin{Verbatim}[commandchars=\\\{\}]
{\color{incolor}In [{\color{incolor}5}]:} \PY{n}{np}\PY{o}{.}\PY{n}{random}\PY{o}{.}\PY{n}{seed}\PY{p}{(}\PY{l+m+mi}{2}\PY{p}{)}
        
        \PY{c+c1}{\PYZsh{} generate 100 x values from 0 to 2 randomly, then sort them in ascending order}
        \PY{n}{X} \PY{o}{=} \PY{l+m+mi}{2} \PY{o}{*} \PY{n}{np}\PY{o}{.}\PY{n}{random}\PY{o}{.}\PY{n}{rand}\PY{p}{(}\PY{l+m+mi}{100}\PY{p}{,} \PY{l+m+mi}{1}\PY{p}{)}
        \PY{n}{X}\PY{o}{.}\PY{n}{sort}\PY{p}{(}\PY{n}{axis}\PY{o}{=}\PY{l+m+mi}{0}\PY{p}{)}
        
        \PY{c+c1}{\PYZsh{} generate y values and add noise to it}
        \PY{n}{y} \PY{o}{=} \PY{l+m+mi}{4} \PY{o}{+} \PY{l+m+mi}{3} \PY{o}{*} \PY{n}{X} \PY{o}{+} \PY{n}{np}\PY{o}{.}\PY{n}{random}\PY{o}{.}\PY{n}{rand}\PY{p}{(}\PY{l+m+mi}{100}\PY{p}{,} \PY{l+m+mi}{1}\PY{p}{)}
        
        \PY{c+c1}{\PYZsh{} let us plot the data}
        \PY{n}{plt}\PY{o}{.}\PY{n}{scatter}\PY{p}{(}\PY{n}{X}\PY{p}{,} \PY{n}{y}\PY{p}{)}
\end{Verbatim}


\begin{Verbatim}[commandchars=\\\{\}]
{\color{outcolor}Out[{\color{outcolor}5}]:} <matplotlib.collections.PathCollection at 0x269541d7fd0>
\end{Verbatim}
            
    \begin{center}
    \adjustimage{max size={0.9\linewidth}{0.9\paperheight}}{output_12_1.png}
    \end{center}
    { \hspace*{\fill} \\}
    
    \begin{Verbatim}[commandchars=\\\{\}]
{\color{incolor}In [{\color{incolor}8}]:} \PY{c+c1}{\PYZsh{} combine the x and y values into a single array called points}
        \PY{n}{points} \PY{o}{=} \PY{n}{np}\PY{o}{.}\PY{n}{column\PYZus{}stack}\PY{p}{(}\PY{p}{(}\PY{n}{X}\PY{p}{,} \PY{n}{y}\PY{p}{)}\PY{p}{)}
        
        \PY{n}{num\PYZus{}iterations} \PY{o}{=} \PY{l+m+mi}{100}
        \PY{n}{learning\PYZus{}rate} \PY{o}{=} \PY{l+m+mf}{0.1}
        \PY{n}{initial\PYZus{}w0} \PY{o}{=} \PY{l+m+mi}{0} \PY{c+c1}{\PYZsh{} initial y\PYZhy{}intercept guess}
        \PY{n}{initial\PYZus{}w1} \PY{o}{=} \PY{l+m+mi}{0} \PY{c+c1}{\PYZsh{} initial slope guess}
        \PY{n}{early\PYZus{}stop} \PY{o}{=} \PY{l+m+mf}{0.00064}
        \PY{p}{[}\PY{n}{w0}\PY{p}{,} \PY{n}{w1}\PY{p}{,} \PY{n}{mse}\PY{p}{]} \PY{o}{=} \PY{n}{gradient\PYZus{}descent\PYZus{}runner\PYZus{}early\PYZus{}stop}\PY{p}{(}\PY{n}{points}\PY{p}{,} \PY{n}{initial\PYZus{}w0}\PY{p}{,} \PY{n}{initial\PYZus{}w1}\PY{p}{,} \PY{n}{learning\PYZus{}rate}\PY{p}{,} \PY{n}{num\PYZus{}iterations}\PY{p}{,} \PY{n}{early\PYZus{}stop}\PY{p}{)}
\end{Verbatim}


    \begin{Verbatim}[commandchars=\\\{\}]
Iteration 1: w0=1.44691, w1=1.50194, mse=20.14308
Iteration 2: w0=2.32806, w1=2.39849, mse=7.47831
Iteration 3: w0=2.86801, w1=2.93046, mse=2.87255
Iteration 4: w0=3.20208, w1=3.24295, mse=1.19394
Iteration 5: w0=3.41183, w1=3.42341, mse=0.57873
Iteration 6: w0=3.54643, w1=3.52453, mse=0.35003
Iteration 7: w0=3.63550, w1=3.57805, mse=0.26199
Iteration 8: w0=3.69691, w1=3.60309, mse=0.22532
Iteration 9: w0=3.74143, w1=3.61118, mse=0.20757
Iteration 10: w0=3.77555, w1=3.60925, mse=0.19693
Iteration 11: w0=3.80321, w1=3.60148, mse=0.18912
Iteration 12: w0=3.82676, w1=3.59037, mse=0.18255
Iteration 13: w0=3.84765, w1=3.57743, mse=0.17664
Iteration 14: w0=3.86674, w1=3.56358, mse=0.17117
Iteration 15: w0=3.88456, w1=3.54934, mse=0.16604
Iteration 16: w0=3.90144, w1=3.53503, mse=0.16122
Iteration 17: w0=3.91758, w1=3.52085, mse=0.15667
Iteration 18: w0=3.93309, w1=3.50690, mse=0.15238
Iteration 19: w0=3.94807, w1=3.49325, mse=0.14833
Iteration 20: w0=3.96257, w1=3.47992, mse=0.14451
Iteration 21: w0=3.97662, w1=3.46694, mse=0.14090
Iteration 22: w0=3.99025, w1=3.45430, mse=0.13750
Iteration 23: w0=4.00348, w1=3.44201, mse=0.13428
Iteration 24: w0=4.01632, w1=3.43005, mse=0.13125
Iteration 25: w0=4.02880, w1=3.41844, mse=0.12838
Iteration 26: w0=4.04092, w1=3.40715, mse=0.12568
Iteration 27: w0=4.05269, w1=3.39618, mse=0.12313
Iteration 28: w0=4.06412, w1=3.38552, mse=0.12072
Iteration 29: w0=4.07523, w1=3.37516, mse=0.11844
Iteration 30: w0=4.08603, w1=3.36510, mse=0.11630
Iteration 31: w0=4.09651, w1=3.35533, mse=0.11427
Iteration 32: w0=4.10670, w1=3.34583, mse=0.11236
Iteration 33: w0=4.11660, w1=3.33660, mse=0.11055
Iteration 34: w0=4.12622, w1=3.32763, mse=0.10885
Iteration 35: w0=4.13556, w1=3.31892, mse=0.10724
Iteration 36: w0=4.14464, w1=3.31046, mse=0.10572
Iteration 37: w0=4.15346, w1=3.30223, mse=0.10429
Iteration 38: w0=4.16203, w1=3.29424, mse=0.10293
Iteration 39: w0=4.17035, w1=3.28648, mse=0.10166
Iteration 40: w0=4.17844, w1=3.27894, mse=0.10045
Iteration 41: w0=4.18630, w1=3.27162, mse=0.09931
Iteration 42: w0=4.19393, w1=3.26450, mse=0.09824
Iteration 43: w0=4.20135, w1=3.25758, mse=0.09722
Iteration 44: w0=4.20856, w1=3.25086, mse=0.09627
Iteration 45: w0=4.21556, w1=3.24433, mse=0.09536
Iteration 46: w0=4.22236, w1=3.23799, mse=0.09451
Iteration 47: w0=4.22897, w1=3.23183, mse=0.09371
Iteration 48: w0=4.23539, w1=3.22584, mse=0.09295
Iteration 49: w0=4.24163, w1=3.22002, mse=0.09223
Iteration 50: w0=4.24769, w1=3.21437, mse=0.09155

    \end{Verbatim}

    \subsection{Exercise 1.1 }\label{exercise-1.1}

    \begin{Verbatim}[commandchars=\\\{\}]
{\color{incolor}In [{\color{incolor}7}]:} \PY{c+c1}{\PYZsh{} TODO:}
        \PY{c+c1}{\PYZsh{} Paste your code below for the modified gradient\PYZus{}descent\PYZus{}runner\PYZus{}early\PYZus{}stop function}
        \PY{k}{def} \PY{n+nf}{gradient\PYZus{}descent\PYZus{}runner\PYZus{}early\PYZus{}stop}\PY{p}{(}\PY{n}{points}\PY{p}{,} \PY{n}{starting\PYZus{}w0}\PY{p}{,} \PY{n}{starting\PYZus{}w1}\PY{p}{,} \PY{n}{learning\PYZus{}rate}\PY{p}{,} \PY{n}{num\PYZus{}iterations}\PY{p}{,} \PY{n}{early\PYZus{}stop}\PY{p}{)}\PY{p}{:}
            \PY{n}{w0} \PY{o}{=} \PY{n}{starting\PYZus{}w0}
            \PY{n}{w1} \PY{o}{=} \PY{n}{starting\PYZus{}w1}
            \PY{n}{mse\PYZus{}t0} \PY{o}{=} \PY{l+m+mf}{1e4} \PY{c+c1}{\PYZsh{} Set to a high number for case 0}
            \PY{k}{for} \PY{n}{i} \PY{o+ow}{in} \PY{n+nb}{range}\PY{p}{(}\PY{n}{num\PYZus{}iterations}\PY{p}{)}\PY{p}{:}
                \PY{n}{w0}\PY{p}{,} \PY{n}{w1} \PY{o}{=} \PY{n}{step\PYZus{}gradient}\PY{p}{(}\PY{n}{w0}\PY{p}{,} \PY{n}{w1}\PY{p}{,} \PY{n}{points}\PY{p}{,} \PY{n}{learning\PYZus{}rate}\PY{p}{)}
                \PY{n}{mse\PYZus{}t1} \PY{o}{=} \PY{n}{compute\PYZus{}error\PYZus{}for\PYZus{}line\PYZus{}given\PYZus{}points}\PY{p}{(}\PY{n}{w0}\PY{p}{,} \PY{n}{w1}\PY{p}{,} \PY{n}{points}\PY{p}{)}
                \PY{k}{if}\PY{p}{(}\PY{n}{mse\PYZus{}t0} \PY{o}{\PYZhy{}} \PY{n}{mse\PYZus{}t1} \PY{o}{\PYZlt{}} \PY{n}{early\PYZus{}stop}\PY{p}{)}\PY{p}{:}
                    \PY{k}{break}\PY{p}{;}
                \PY{k}{else}\PY{p}{:}
                    \PY{n+nb}{print}\PY{p}{(}\PY{n}{f}\PY{l+s+s1}{\PYZsq{}}\PY{l+s+s1}{Iteration }\PY{l+s+s1}{\PYZob{}}\PY{l+s+s1}{i+1\PYZcb{}: w0=}\PY{l+s+si}{\PYZob{}w0:0.5f\PYZcb{}}\PY{l+s+s1}{, w1=}\PY{l+s+si}{\PYZob{}w1:0.5f\PYZcb{}}\PY{l+s+s1}{, mse=}\PY{l+s+si}{\PYZob{}mse\PYZus{}t1:0.5f\PYZcb{}}\PY{l+s+s1}{\PYZsq{}}\PY{p}{)}
                    \PY{n}{mse\PYZus{}t0} \PY{o}{=} \PY{n}{mse\PYZus{}t1}
            \PY{k}{return} \PY{p}{[}\PY{n}{w0}\PY{p}{,} \PY{n}{w1}\PY{p}{,} \PY{n}{mse\PYZus{}t0}\PY{p}{]}
        \PY{c+c1}{\PYZsh{} N.B.: Students who wish to do this exercise in R should implement this function in R themselves}
\end{Verbatim}


    \subsection{Exercise 1.2}\label{exercise-1.2}

In the program above we had set the learning rate to 0.1. Using the
original \texttt{gradient\_descent\_runner} function, first set the
number of iterations to 100. Then try to run the code with two different
values of learning rates: 1. a learning rate of 0.001 2. a learning rate
of 1

Explain what you observe.

    \begin{Verbatim}[commandchars=\\\{\}]
{\color{incolor}In [{\color{incolor}9}]:} \PY{o}{\PYZpc{}\PYZpc{}}\PY{k}{capture}
        num\PYZus{}iterations = 100
        learning\PYZus{}rate = [1, 1e\PYZhy{}3]
        initial\PYZus{}w0 = 0 \PYZsh{} initial y\PYZhy{}intercept guess
        initial\PYZus{}w1 = 0 \PYZsh{} initial slope guess
        for lr in learning\PYZus{}rate:
            print(\PYZdq{}\PYZhy{}\PYZhy{}\PYZhy{}\PYZhy{}\PYZhy{}\PYZhy{}\PYZhy{}\PYZhy{}\PYZhy{}\PYZhy{}\PYZhy{}\PYZhy{}\PYZhy{}\PYZhy{}\PYZhy{}\PYZhy{}\PYZhy{}\PYZhy{}\PYZhy{}\PYZhy{}\PYZhy{}\PYZdq{} + \PYZdq{}\PYZbs{}n\PYZdq{} + \PYZdq{}Learning rate: \PYZdq{} + str(lr))
            [w0, w1, mse] = gradient\PYZus{}descent\PYZus{}runner(points, initial\PYZus{}w0, initial\PYZus{}w1, lr, num\PYZus{}iterations)
\end{Verbatim}


    \subparagraph{TODO}\label{todo}

\begin{quote}
When learning rate is set to 1 the mse (drastically) increases. When
learning rate is set to 0.001 the mse decreases, however it decreases
slowly. Conclusion: Learning rate = 1 is too high, while learning rate =
0.001 is too low.
\end{quote}

    \subsection{Exercise 1.3}\label{exercise-1.3}

    \begin{Verbatim}[commandchars=\\\{\}]
{\color{incolor}In [{\color{incolor}10}]:} \PY{c+c1}{\PYZsh{} TODO}
         \PY{c+c1}{\PYZsh{} Write your solution here}
         \PY{n}{shape} \PY{o}{=} \PY{n}{X}\PY{o}{.}\PY{n}{shape}
         \PY{n}{obj\PYZus{}type} \PY{o}{=} \PY{n}{X}\PY{o}{.}\PY{n}{dtype}
         \PY{n}{ones} \PY{o}{=} \PY{n}{np}\PY{o}{.}\PY{n}{ones}\PY{p}{(}\PY{p}{(}\PY{n}{shape}\PY{p}{)}\PY{p}{,} \PY{n}{obj\PYZus{}type}\PY{p}{)}
         \PY{n}{X\PYZus{}ones} \PY{o}{=} \PY{n}{np}\PY{o}{.}\PY{n}{hstack}\PY{p}{(}\PY{p}{(}\PY{n}{X}\PY{p}{,} \PY{n}{ones}\PY{p}{)}\PY{p}{)}
\end{Verbatim}


    \begin{Verbatim}[commandchars=\\\{\}]
{\color{incolor}In [{\color{incolor}11}]:} \PY{n}{X\PYZus{}trans} \PY{o}{=} \PY{n}{X\PYZus{}ones}\PY{o}{.}\PY{n}{T}
\end{Verbatim}


    \begin{Verbatim}[commandchars=\\\{\}]
{\color{incolor}In [{\color{incolor}12}]:} \PY{n}{X\PYZus{}trans\PYZus{}times\PYZus{}X} \PY{o}{=} \PY{n}{X\PYZus{}trans}\PY{o}{.}\PY{n}{dot}\PY{p}{(}\PY{n}{X\PYZus{}ones}\PY{p}{)}
\end{Verbatim}


    \begin{Verbatim}[commandchars=\\\{\}]
{\color{incolor}In [{\color{incolor}13}]:} \PY{k+kn}{from} \PY{n+nn}{numpy}\PY{n+nn}{.}\PY{n+nn}{linalg} \PY{k}{import} \PY{n}{inv}
\end{Verbatim}


    \begin{Verbatim}[commandchars=\\\{\}]
{\color{incolor}In [{\color{incolor}14}]:} \PY{n}{X\PYZus{}T\PYZus{}times\PYZus{}X\PYZus{}\PYZus{}inv} \PY{o}{=} \PY{n}{inv}\PY{p}{(}\PY{n}{X\PYZus{}trans\PYZus{}times\PYZus{}X}\PY{p}{)}
\end{Verbatim}


    \begin{Verbatim}[commandchars=\\\{\}]
{\color{incolor}In [{\color{incolor}15}]:} \PY{n}{w\PYZus{}hat} \PY{o}{=} \PY{p}{(}\PY{p}{(}\PY{n}{X\PYZus{}T\PYZus{}times\PYZus{}X\PYZus{}\PYZus{}inv}\PY{o}{.}\PY{n}{dot}\PY{p}{(}\PY{n}{X\PYZus{}trans}\PY{p}{)}\PY{p}{)}\PY{o}{.}\PY{n}{dot}\PY{p}{(}\PY{n}{y}\PY{p}{)}\PY{p}{)}
\end{Verbatim}


    \begin{Verbatim}[commandchars=\\\{\}]
{\color{incolor}In [{\color{incolor}16}]:} \PY{n}{w\PYZus{}hat}
\end{Verbatim}


\begin{Verbatim}[commandchars=\\\{\}]
{\color{outcolor}Out[{\color{outcolor}16}]:} array([[3.02129039],
                [4.45478709]])
\end{Verbatim}
            
    \begin{quote}
It seems like the values are more or less the same, so both methods
obviously work.
\end{quote}

    \subsection{Exercise 1.4}\label{exercise-1.4}

To brush up your calculus skills, derive the partial derivate of MSE
that has \emph{L2} penalty term included in it. In other words, we want
you compute the following partial derivatives: \[
\frac{\partial}{\partial w_0} \left(\dfrac{1}{n} \sum_{i=1}^{n} [y_i - (w_1 x_i + w_0)]^2 + \lambda {w_0}^2\right) 
\]

\[
\frac{\partial}{\partial w_1} \left(\dfrac{1}{n} \sum_{i=1}^{n} [y_i - (w_1 x_i + w_0)]^2 + \lambda {w_1}^2\right) 
\]

\paragraph{What to submit}\label{what-to-submit}

A derivation of both the gradient equations.

    \[
\frac{\partial}{\partial w_0} \left(\dfrac{1}{n} \sum_{i=1}^{n} [y_i - (w_1 x_i + w_0)]^2  + \lambda \cdot {w_0}^2 \right) =
\underline{-\dfrac{2}{n}\sum_{i=1}^{n} \left(y_i - (w_1 x_i + w_0) + 2 \lambda w_0 \right)}
\]

\[
\frac{\partial}{\partial w_1} \left(\dfrac{1}{n} \sum_{i=1}^{n} [y_i - (w_1 x_i + w_0)]^2  + \lambda \cdot {w_1}^2 \right) =
\underline{-\dfrac{2}{n} \sum_{i=1}^{n} x_i\left (y_i - (w_1 x_i + w_0)\right) + 2 \lambda w_1}
\]

    \section{\texorpdfstring{Linear Regression with \emph{sklearn} Machine
Learning
Library}{Linear Regression with sklearn Machine Learning Library}}\label{linear-regression-with-sklearn-machine-learning-library}

    \begin{Verbatim}[commandchars=\\\{\}]
{\color{incolor}In [{\color{incolor}17}]:} \PY{c+c1}{\PYZsh{} import LinearRegression class from sklearn.linear\PYZus{}model module}
         \PY{k+kn}{from} \PY{n+nn}{sklearn}\PY{n+nn}{.}\PY{n+nn}{linear\PYZus{}model} \PY{k}{import} \PY{n}{LinearRegression}
         
         \PY{c+c1}{\PYZsh{} make a lin\PYZus{}reg object form the LinearRegression class}
         \PY{n}{lin\PYZus{}reg} \PY{o}{=} \PY{n}{LinearRegression}\PY{p}{(}\PY{p}{)}
         
         \PY{c+c1}{\PYZsh{} use the fit method of LinearRegression class to fit a straight line through the data}
         \PY{n}{lin\PYZus{}reg}\PY{o}{.}\PY{n}{fit}\PY{p}{(}\PY{n}{X}\PY{p}{,} \PY{n}{y}\PY{p}{)}
\end{Verbatim}


\begin{Verbatim}[commandchars=\\\{\}]
{\color{outcolor}Out[{\color{outcolor}17}]:} LinearRegression(copy\_X=True, fit\_intercept=True, n\_jobs=1, normalize=False)
\end{Verbatim}
            
    \begin{Verbatim}[commandchars=\\\{\}]
{\color{incolor}In [{\color{incolor}18}]:} \PY{n+nb}{print}\PY{p}{(}\PY{l+s+s1}{\PYZsq{}}\PY{l+s+s1}{slope w1:}\PY{l+s+s1}{\PYZsq{}}\PY{p}{,} \PY{n}{lin\PYZus{}reg}\PY{o}{.}\PY{n}{coef\PYZus{}}\PY{p}{)}
         \PY{n+nb}{print}\PY{p}{(}\PY{l+s+s1}{\PYZsq{}}\PY{l+s+s1}{y\PYZhy{}intercept w0:}\PY{l+s+s1}{\PYZsq{}}\PY{p}{,} \PY{n}{lin\PYZus{}reg}\PY{o}{.}\PY{n}{intercept\PYZus{}}\PY{p}{)}
\end{Verbatim}


    \begin{Verbatim}[commandchars=\\\{\}]
slope w1: [[3.02129039]]
y-intercept w0: [4.45478709]

    \end{Verbatim}

    \begin{Verbatim}[commandchars=\\\{\}]
{\color{incolor}In [{\color{incolor}19}]:} \PY{c+c1}{\PYZsh{} plot the original data points as a scatter plot}
         \PY{n}{plt}\PY{o}{.}\PY{n}{scatter}\PY{p}{(}\PY{n}{X}\PY{p}{,} \PY{n}{y}\PY{p}{,} \PY{n}{label}\PY{o}{=}\PY{l+s+s1}{\PYZsq{}}\PY{l+s+s1}{original data}\PY{l+s+s1}{\PYZsq{}}\PY{p}{)}
         
         \PY{c+c1}{\PYZsh{} plot the line that fits these points. Use the values of m and b as provided by the fit method}
         \PY{n}{y\PYZus{}} \PY{o}{=} \PY{n}{lin\PYZus{}reg}\PY{o}{.}\PY{n}{coef\PYZus{}}\PY{o}{*}\PY{n}{X} \PY{o}{+} \PY{n}{lin\PYZus{}reg}\PY{o}{.}\PY{n}{intercept\PYZus{}}
         
         \PY{c+c1}{\PYZsh{} you can also get y\PYZus{} by using the predict method. Uncomment the line below:}
         \PY{c+c1}{\PYZsh{}y\PYZus{} = lin\PYZus{}reg.predict(X)}
         
         \PY{n}{plt}\PY{o}{.}\PY{n}{plot}\PY{p}{(}\PY{n}{X}\PY{p}{,} \PY{n}{y\PYZus{}}\PY{p}{,} \PY{n}{color}\PY{o}{=}\PY{l+s+s1}{\PYZsq{}}\PY{l+s+s1}{r}\PY{l+s+s1}{\PYZsq{}}\PY{p}{,} \PY{n}{label}\PY{o}{=}\PY{l+s+s1}{\PYZsq{}}\PY{l+s+s1}{predicted fit}\PY{l+s+s1}{\PYZsq{}}\PY{p}{)}
         \PY{n}{plt}\PY{o}{.}\PY{n}{xlabel}\PY{p}{(}\PY{l+s+s1}{\PYZsq{}}\PY{l+s+s1}{x}\PY{l+s+s1}{\PYZsq{}}\PY{p}{)}\PY{p}{;} \PY{n}{plt}\PY{o}{.}\PY{n}{ylabel}\PY{p}{(}\PY{l+s+s1}{\PYZsq{}}\PY{l+s+s1}{y}\PY{l+s+s1}{\PYZsq{}}\PY{p}{)}
         \PY{n}{plt}\PY{o}{.}\PY{n}{legend}\PY{p}{(}\PY{n}{loc}\PY{o}{=}\PY{l+s+s1}{\PYZsq{}}\PY{l+s+s1}{best}\PY{l+s+s1}{\PYZsq{}}\PY{p}{)}
\end{Verbatim}


\begin{Verbatim}[commandchars=\\\{\}]
{\color{outcolor}Out[{\color{outcolor}19}]:} <matplotlib.legend.Legend at 0x26955ba9630>
\end{Verbatim}
            
    \begin{center}
    \adjustimage{max size={0.9\linewidth}{0.9\paperheight}}{output_33_1.png}
    \end{center}
    { \hspace*{\fill} \\}
    
    \subsection{Exercise 1.5}\label{exercise-1.5}

    \begin{enumerate}
\def\labelenumi{\arabic{enumi}.}
\tightlist
\item
  The \emph{fit} method from the \textbf{sklearn library} seems to give
  more or less the same values for \emph{b} and \emph{m} as "my"
  implementation of the \textbf{Normal Equation}
\end{enumerate}

    \begin{Verbatim}[commandchars=\\\{\}]
{\color{incolor}In [{\color{incolor}20}]:} \PY{c+c1}{\PYZsh{} Answer to 2. }
         \PY{n}{test\PYZus{}data\PYZus{}point} \PY{o}{=} \PY{l+m+mi}{3}
         \PY{n}{pred\PYZus{}value} \PY{o}{=} \PY{n+nb}{float}\PY{p}{(}\PY{n}{lin\PYZus{}reg}\PY{o}{.}\PY{n}{predict}\PY{p}{(}\PY{n}{test\PYZus{}data\PYZus{}point}\PY{p}{)}\PY{p}{)}
         \PY{n}{actual\PYZus{}value} \PY{o}{=} \PY{l+m+mi}{4} \PY{o}{+} \PY{l+m+mi}{3}\PY{o}{*}\PY{n}{test\PYZus{}data\PYZus{}point}
         \PY{n+nb}{print}\PY{p}{(}\PY{l+s+s2}{\PYZdq{}}\PY{l+s+s2}{Predicted value: }\PY{l+s+s2}{\PYZdq{}} \PY{o}{+} \PY{n+nb}{str}\PY{p}{(}\PY{n}{pred\PYZus{}value}\PY{p}{)} \PY{o}{+} \PY{l+s+s1}{\PYZsq{}}\PY{l+s+se}{\PYZbs{}n}\PY{l+s+s1}{\PYZsq{}} \PY{o}{+} \PY{l+s+s2}{\PYZdq{}}\PY{l+s+s2}{Actual value: }\PY{l+s+s2}{\PYZdq{}} \PY{o}{+} \PY{n+nb}{str}\PY{p}{(}\PY{n}{actual\PYZus{}value}\PY{p}{)}\PY{p}{)}
\end{Verbatim}


    \begin{Verbatim}[commandchars=\\\{\}]
Predicted value: 13.518658256328946
Actual value: 13

    \end{Verbatim}

    \begin{quote}
Pretty fair prediction
\end{quote}

    \section{2. Multivariate Linear Regression
}\label{multivariate-linear-regression}

    \begin{Verbatim}[commandchars=\\\{\}]
{\color{incolor}In [{\color{incolor}21}]:} \PY{c+c1}{\PYZsh{} make a dataframe of the data}
         \PY{n}{df} \PY{o}{=} \PY{n}{pd}\PY{o}{.}\PY{n}{read\PYZus{}csv}\PY{p}{(}\PY{l+s+s1}{\PYZsq{}}\PY{l+s+s1}{movies.csv}\PY{l+s+s1}{\PYZsq{}}\PY{p}{)}
         
         \PY{c+c1}{\PYZsh{} show first five rows of df}
         \PY{n}{df}\PY{o}{.}\PY{n}{head}\PY{p}{(}\PY{n}{n}\PY{o}{=}\PY{l+m+mi}{5}\PY{p}{)}
\end{Verbatim}


\begin{Verbatim}[commandchars=\\\{\}]
{\color{outcolor}Out[{\color{outcolor}21}]:}       revenue  production\_cost  promotional\_cost  book\_sales
         0   85.099998              8.5          5.100000         4.7
         1  106.300003             12.9          5.800000         8.8
         2   50.200001              5.2          2.100000        15.1
         3  130.600006             10.7          8.399999        12.2
         4   54.799999              3.1          2.900000        10.6
\end{Verbatim}
            
    \begin{Verbatim}[commandchars=\\\{\}]
{\color{incolor}In [{\color{incolor}22}]:} \PY{c+c1}{\PYZsh{} Extract the first column and set it to the output or dependent varaible y}
         \PY{n}{y} \PY{o}{=} \PY{n}{df}\PY{p}{[}\PY{p}{[}\PY{l+s+s1}{\PYZsq{}}\PY{l+s+s1}{revenue}\PY{l+s+s1}{\PYZsq{}}\PY{p}{]}\PY{p}{]}
         
         \PY{c+c1}{\PYZsh{} Remove the first column and set the rest of the dataframe to X. This is the set of indepedent variables}
         \PY{n}{X} \PY{o}{=} \PY{n}{df}\PY{o}{.}\PY{n}{drop}\PY{p}{(}\PY{n}{columns}\PY{o}{=}\PY{p}{[}\PY{l+s+s1}{\PYZsq{}}\PY{l+s+s1}{revenue}\PY{l+s+s1}{\PYZsq{}}\PY{p}{]}\PY{p}{)}
         
         \PY{c+c1}{\PYZsh{} show first five rows of X}
         \PY{n}{X}\PY{o}{.}\PY{n}{head}\PY{p}{(}\PY{n}{n}\PY{o}{=}\PY{l+m+mi}{5}\PY{p}{)}
\end{Verbatim}


\begin{Verbatim}[commandchars=\\\{\}]
{\color{outcolor}Out[{\color{outcolor}22}]:}    production\_cost  promotional\_cost  book\_sales
         0              8.5          5.100000         4.7
         1             12.9          5.800000         8.8
         2              5.2          2.100000        15.1
         3             10.7          8.399999        12.2
         4              3.1          2.900000        10.6
\end{Verbatim}
            
    \begin{Verbatim}[commandchars=\\\{\}]
{\color{incolor}In [{\color{incolor}23}]:} \PY{c+c1}{\PYZsh{} show first five rows of y}
         \PY{n}{y}\PY{o}{.}\PY{n}{head}\PY{p}{(}\PY{n}{n}\PY{o}{=}\PY{l+m+mi}{5}\PY{p}{)}
\end{Verbatim}


\begin{Verbatim}[commandchars=\\\{\}]
{\color{outcolor}Out[{\color{outcolor}23}]:}       revenue
         0   85.099998
         1  106.300003
         2   50.200001
         3  130.600006
         4   54.799999
\end{Verbatim}
            
    \begin{Verbatim}[commandchars=\\\{\}]
{\color{incolor}In [{\color{incolor}24}]:} \PY{k+kn}{from} \PY{n+nn}{sklearn}\PY{n+nn}{.}\PY{n+nn}{linear\PYZus{}model} \PY{k}{import} \PY{n}{LinearRegression}
         
         \PY{c+c1}{\PYZsh{} make a lin\PYZus{}reg object form the LinearRegression class}
         \PY{n}{lin\PYZus{}reg} \PY{o}{=} \PY{n}{LinearRegression}\PY{p}{(}\PY{p}{)}
         
         \PY{c+c1}{\PYZsh{} use the fit method of LinearRegression class to fit a straight line through the data}
         \PY{n}{lin\PYZus{}reg}\PY{o}{.}\PY{n}{fit}\PY{p}{(}\PY{n}{X}\PY{p}{,} \PY{n}{y}\PY{p}{)}
         
         \PY{c+c1}{\PYZsh{} Display the learned parameters}
         \PY{n}{lin\PYZus{}reg}\PY{o}{.}\PY{n}{intercept\PYZus{}}\PY{p}{,} \PY{n}{lin\PYZus{}reg}\PY{o}{.}\PY{n}{coef\PYZus{}}
\end{Verbatim}


\begin{Verbatim}[commandchars=\\\{\}]
{\color{outcolor}Out[{\color{outcolor}24}]:} (array([7.67602854]), array([[3.66160401, 7.62105126, 0.82846807]]))
\end{Verbatim}
            
    \subsection{Exercise 2.1}\label{exercise-2.1}

    \begin{Verbatim}[commandchars=\\\{\}]
{\color{incolor}In [{\color{incolor}25}]:} \PY{c+c1}{\PYZsh{}\PYZsh{} TODO}
         \PY{c+c1}{\PYZsh{}\PYZsh{} Write your code here}
         \PY{n}{prod\PYZus{}c} \PY{o}{=} \PY{l+m+mi}{23}
         \PY{n}{prom\PYZus{}c} \PY{o}{=} \PY{l+m+mi}{12}
         \PY{n}{book\PYZus{}s} \PY{o}{=} \PY{l+m+mi}{10}
         \PY{n}{X\PYZus{}test} \PY{o}{=} \PY{n}{np}\PY{o}{.}\PY{n}{array}\PY{p}{(}\PY{p}{[}\PY{n}{prod\PYZus{}c}\PY{p}{,} \PY{n}{prom\PYZus{}c}\PY{p}{,} \PY{n}{book\PYZus{}s}\PY{p}{]}\PY{p}{)}\PY{o}{.}\PY{n}{reshape}\PY{p}{(}\PY{l+m+mi}{1}\PY{p}{,} \PY{o}{\PYZhy{}}\PY{l+m+mi}{1}\PY{p}{)}
         \PY{n}{lin\PYZus{}reg}\PY{o}{.}\PY{n}{predict}\PY{p}{(}\PY{n}{X\PYZus{}test}\PY{p}{)}
\end{Verbatim}


\begin{Verbatim}[commandchars=\\\{\}]
{\color{outcolor}Out[{\color{outcolor}25}]:} array([[191.6302165]])
\end{Verbatim}
            
    \section{Multivariate Regression with Polynomial
basis}\label{multivariate-regression-with-polynomial-basis}

    \begin{Verbatim}[commandchars=\\\{\}]
{\color{incolor}In [{\color{incolor}26}]:} \PY{c+c1}{\PYZsh{} define the number of points to generate as k}
         \PY{n}{k} \PY{o}{=} \PY{l+m+mi}{100}
         
         \PY{c+c1}{\PYZsh{} define a seed value. It is important to define the seed value}
         \PY{c+c1}{\PYZsh{} so that the random numbers generated are the same every time}
         \PY{c+c1}{\PYZsh{} this code is executed.}
         \PY{n}{np}\PY{o}{.}\PY{n}{random}\PY{o}{.}\PY{n}{seed}\PY{p}{(}\PY{l+m+mi}{10}\PY{p}{)}
         
         \PY{c+c1}{\PYZsh{} generate k x\PYZhy{}axis values from \PYZhy{}3 to +3}
         \PY{n}{X} \PY{o}{=} \PY{l+m+mi}{6} \PY{o}{*} \PY{n}{np}\PY{o}{.}\PY{n}{random}\PY{o}{.}\PY{n}{rand}\PY{p}{(}\PY{n}{k}\PY{p}{,} \PY{l+m+mi}{1}\PY{p}{)} \PY{o}{\PYZhy{}} \PY{l+m+mi}{3}
         
         \PY{c+c1}{\PYZsh{} sort the numbers in ascending order. This helps when we are plotting the data. }
         \PY{c+c1}{\PYZsh{} Without this line, your plots will be all jumbled up}
         \PY{n}{X}\PY{o}{.}\PY{n}{sort}\PY{p}{(}\PY{n}{axis}\PY{o}{=}\PY{l+m+mi}{0}\PY{p}{)}
         
         \PY{c+c1}{\PYZsh{} generate k y\PYZhy{}axis values}
         \PY{n}{y} \PY{o}{=} \PY{l+m+mf}{0.5} \PY{o}{*} \PY{n}{X}\PY{o}{*}\PY{o}{*}\PY{l+m+mi}{2} \PY{o}{+} \PY{n}{X} \PY{o}{+} \PY{l+m+mi}{2} \PY{o}{+} \PY{n}{np}\PY{o}{.}\PY{n}{random}\PY{o}{.}\PY{n}{rand}\PY{p}{(}\PY{n}{k}\PY{p}{,} \PY{l+m+mi}{1}\PY{p}{)}
\end{Verbatim}


    Let us now plot the data:

    \begin{Verbatim}[commandchars=\\\{\}]
{\color{incolor}In [{\color{incolor}27}]:} \PY{n}{plt}\PY{o}{.}\PY{n}{scatter}\PY{p}{(}\PY{n}{X}\PY{p}{,} \PY{n}{y}\PY{p}{)}
         \PY{n}{plt}\PY{o}{.}\PY{n}{xlabel}\PY{p}{(}\PY{l+s+s1}{\PYZsq{}}\PY{l+s+s1}{x}\PY{l+s+s1}{\PYZsq{}}\PY{p}{)}\PY{p}{;} \PY{n}{plt}\PY{o}{.}\PY{n}{ylabel}\PY{p}{(}\PY{l+s+s1}{\PYZsq{}}\PY{l+s+s1}{y}\PY{l+s+s1}{\PYZsq{}}\PY{p}{)}
\end{Verbatim}


\begin{Verbatim}[commandchars=\\\{\}]
{\color{outcolor}Out[{\color{outcolor}27}]:} Text(0,0.5,'y')
\end{Verbatim}
            
    \begin{center}
    \adjustimage{max size={0.9\linewidth}{0.9\paperheight}}{output_48_1.png}
    \end{center}
    { \hspace*{\fill} \\}
    
    \begin{Verbatim}[commandchars=\\\{\}]
{\color{incolor}In [{\color{incolor}28}]:} \PY{k+kn}{from} \PY{n+nn}{sklearn}\PY{n+nn}{.}\PY{n+nn}{preprocessing} \PY{k}{import} \PY{n}{PolynomialFeatures}
         
         \PY{n}{poly\PYZus{}features} \PY{o}{=} \PY{n}{PolynomialFeatures}\PY{p}{(}\PY{n}{degree}\PY{o}{=}\PY{l+m+mi}{15}\PY{p}{,} \PY{n}{include\PYZus{}bias}\PY{o}{=}\PY{k+kc}{False}\PY{p}{)}
         
         \PY{c+c1}{\PYZsh{} generate polyonimal features upto degree 2 from the vector X}
         \PY{n}{X\PYZus{}poly} \PY{o}{=} \PY{n}{poly\PYZus{}features}\PY{o}{.}\PY{n}{fit\PYZus{}transform}\PY{p}{(}\PY{n}{X}\PY{p}{)}
\end{Verbatim}


    \begin{Verbatim}[commandchars=\\\{\}]
{\color{incolor}In [{\color{incolor}29}]:} \PY{c+c1}{\PYZsh{} display 4 original data points }
         \PY{n}{X}\PY{p}{[}\PY{l+m+mi}{1}\PY{p}{:}\PY{l+m+mi}{5}\PY{p}{]}
\end{Verbatim}


\begin{Verbatim}[commandchars=\\\{\}]
{\color{outcolor}Out[{\color{outcolor}29}]:} array([[-2.8754883 ],
                [-2.84760131],
                [-2.7643094 ],
                [-2.76024475]])
\end{Verbatim}
            
    \begin{Verbatim}[commandchars=\\\{\}]
{\color{incolor}In [{\color{incolor}30}]:} \PY{c+c1}{\PYZsh{} Display the transformed data.}
         \PY{c+c1}{\PYZsh{} You will now see the original X data alongside its corresponding 2nd\PYZhy{}degree polynomial feature}
         \PY{n}{X\PYZus{}poly}\PY{p}{[}\PY{l+m+mi}{1}\PY{p}{:}\PY{l+m+mi}{5}\PY{p}{]}
\end{Verbatim}


\begin{Verbatim}[commandchars=\\\{\}]
{\color{outcolor}Out[{\color{outcolor}30}]:} array([[-2.87548830e+00,  8.26843299e+00, -2.37757823e+01,
                  6.83669840e+01, -1.96588463e+02,  5.65287826e+02,
                 -1.62547853e+03,  4.67404451e+03, -1.34401603e+04,
                  3.86470238e+04, -1.11129065e+05,  3.19550326e+05,
                 -9.18863225e+05,  2.64218046e+06, -7.59755900e+06],
                [-2.84760131e+00,  8.10883321e+00, -2.30907240e+01,
                  6.57531760e+01, -1.87238830e+02,  5.33181537e+02,
                 -1.51828844e+03,  4.32348015e+03, -1.23115477e+04,
                  3.50583794e+04, -9.98322871e+04,  2.84282551e+05,
                 -8.09523365e+05,  2.30519979e+06, -6.56428994e+06],
                [-2.76430940e+00,  7.64140644e+00, -2.11232116e+01,
                  5.83910924e+01, -1.61411045e+02,  4.46190069e+02,
                 -1.23340740e+03,  3.40951967e+03, -9.42496726e+03,
                  2.60535256e+04, -7.20200055e+04,  1.99085578e+05,
                 -5.50334134e+05,  1.52129382e+06, -4.20532680e+06],
                [-2.76024475e+00,  7.61895107e+00, -2.10301697e+01,
                  5.80484154e+01, -1.60227834e+02,  4.42268036e+02,
                 -1.22076802e+03,  3.36961853e+03, -9.30097184e+03,
                  2.56729587e+04, -7.08636494e+04,  1.95601016e+05,
                 -5.39906677e+05,  1.49027457e+06, -4.11352255e+06]])
\end{Verbatim}
            
    \begin{Verbatim}[commandchars=\\\{\}]
{\color{incolor}In [{\color{incolor}31}]:} \PY{n}{lin\PYZus{}reg} \PY{o}{=} \PY{n}{LinearRegression}\PY{p}{(}\PY{p}{)}
         
         \PY{c+c1}{\PYZsh{} Now we fit a linear model to the X\PYZus{}poly (the transformed features set) and y}
         \PY{n}{lin\PYZus{}reg}\PY{o}{.}\PY{n}{fit}\PY{p}{(}\PY{n}{X\PYZus{}poly}\PY{p}{,} \PY{n}{y}\PY{p}{)}
         
         \PY{c+c1}{\PYZsh{} show the values of intercept and learned co\PYZhy{}efficients}
         \PY{n}{lin\PYZus{}reg}\PY{o}{.}\PY{n}{intercept\PYZus{}}\PY{p}{,} \PY{n}{lin\PYZus{}reg}\PY{o}{.}\PY{n}{coef\PYZus{}}
\end{Verbatim}


\begin{Verbatim}[commandchars=\\\{\}]
{\color{outcolor}Out[{\color{outcolor}31}]:} (array([2.43771889]),
          array([[ 4.70240062e-01,  9.81483416e-01,  1.45901724e+00,
                  -5.43920910e-01, -1.30913386e+00,  2.30186856e-01,
                   5.62777734e-01, -4.49943278e-02, -1.28760085e-01,
                   3.53201168e-03,  1.58848833e-02,  1.15315775e-05,
                  -9.84782281e-04, -1.01041607e-05,  2.36325364e-05]]))
\end{Verbatim}
            
    \begin{Verbatim}[commandchars=\\\{\}]
{\color{incolor}In [{\color{incolor}32}]:} \PY{n}{y\PYZus{}} \PY{o}{=} \PY{n}{lin\PYZus{}reg}\PY{o}{.}\PY{n}{predict}\PY{p}{(}\PY{n}{X\PYZus{}poly}\PY{p}{)}
             
         \PY{n}{plt}\PY{o}{.}\PY{n}{scatter}\PY{p}{(}\PY{n}{X}\PY{p}{,} \PY{n}{y}\PY{p}{,} \PY{n}{label}\PY{o}{=}\PY{l+s+s1}{\PYZsq{}}\PY{l+s+s1}{original data}\PY{l+s+s1}{\PYZsq{}}\PY{p}{)}
         \PY{n}{plt}\PY{o}{.}\PY{n}{plot}\PY{p}{(}\PY{n}{X}\PY{p}{,} \PY{n}{y\PYZus{}}\PY{p}{,} \PY{n}{color}\PY{o}{=}\PY{l+s+s1}{\PYZsq{}}\PY{l+s+s1}{r}\PY{l+s+s1}{\PYZsq{}}\PY{p}{,} \PY{n}{label}\PY{o}{=}\PY{l+s+s1}{\PYZsq{}}\PY{l+s+s1}{predicted fit}\PY{l+s+s1}{\PYZsq{}}\PY{p}{)}
         \PY{n}{plt}\PY{o}{.}\PY{n}{legend}\PY{p}{(}\PY{n}{loc}\PY{o}{=}\PY{l+s+s1}{\PYZsq{}}\PY{l+s+s1}{best}\PY{l+s+s1}{\PYZsq{}}\PY{p}{)}
         
         \PY{n}{plt}\PY{o}{.}\PY{n}{xlabel}\PY{p}{(}\PY{l+s+s1}{\PYZsq{}}\PY{l+s+s1}{x}\PY{l+s+s1}{\PYZsq{}}\PY{p}{)}
         \PY{n}{plt}\PY{o}{.}\PY{n}{ylabel}\PY{p}{(}\PY{l+s+s1}{\PYZsq{}}\PY{l+s+s1}{y}\PY{l+s+s1}{\PYZsq{}}\PY{p}{)}
\end{Verbatim}


\begin{Verbatim}[commandchars=\\\{\}]
{\color{outcolor}Out[{\color{outcolor}32}]:} Text(0,0.5,'y')
\end{Verbatim}
            
    \begin{center}
    \adjustimage{max size={0.9\linewidth}{0.9\paperheight}}{output_53_1.png}
    \end{center}
    { \hspace*{\fill} \\}
    
    \subsubsection{Regularization with Ridge
Penalty}\label{regularization-with-ridge-penalty}

    \begin{Verbatim}[commandchars=\\\{\}]
{\color{incolor}In [{\color{incolor}33}]:} \PY{k+kn}{from} \PY{n+nn}{sklearn}\PY{n+nn}{.}\PY{n+nn}{linear\PYZus{}model} \PY{k}{import} \PY{n}{Ridge}
         \PY{k+kn}{from} \PY{n+nn}{sklearn}\PY{n+nn}{.}\PY{n+nn}{linear\PYZus{}model} \PY{k}{import} \PY{n}{LinearRegression}
         \PY{k+kn}{from} \PY{n+nn}{sklearn}\PY{n+nn}{.}\PY{n+nn}{preprocessing} \PY{k}{import} \PY{n}{PolynomialFeatures}
         
         \PY{c+c1}{\PYZsh{} define the number of points to generate}
         \PY{n}{k} \PY{o}{=} \PY{l+m+mi}{100}
         \PY{n}{np}\PY{o}{.}\PY{n}{random}\PY{o}{.}\PY{n}{seed}\PY{p}{(}\PY{l+m+mi}{10}\PY{p}{)}
         
         \PY{c+c1}{\PYZsh{} generate k x\PYZhy{}axis values from \PYZhy{}3 to +3}
         \PY{n}{X} \PY{o}{=} \PY{l+m+mi}{6} \PY{o}{*} \PY{n}{np}\PY{o}{.}\PY{n}{random}\PY{o}{.}\PY{n}{rand}\PY{p}{(}\PY{n}{k}\PY{p}{,} \PY{l+m+mi}{1}\PY{p}{)} \PY{o}{\PYZhy{}} \PY{l+m+mi}{3}
         \PY{n}{X}\PY{o}{.}\PY{n}{sort}\PY{p}{(}\PY{n}{axis}\PY{o}{=}\PY{l+m+mi}{0}\PY{p}{)}
         
         \PY{c+c1}{\PYZsh{} generate k y\PYZhy{}axis values}
         \PY{n}{y} \PY{o}{=} \PY{l+m+mf}{0.5} \PY{o}{*} \PY{n}{X}\PY{o}{*}\PY{o}{*}\PY{l+m+mi}{2} \PY{o}{+} \PY{n}{X} \PY{o}{+} \PY{l+m+mi}{2} \PY{o}{+} \PY{n}{np}\PY{o}{.}\PY{n}{random}\PY{o}{.}\PY{n}{randn}\PY{p}{(}\PY{n}{k}\PY{p}{,} \PY{l+m+mi}{1}\PY{p}{)}
         
         \PY{c+c1}{\PYZsh{} Create polynomial feature (degree 15)}
         \PY{n}{poly\PYZus{}features} \PY{o}{=} \PY{n}{PolynomialFeatures}\PY{p}{(}\PY{n}{degree}\PY{o}{=}\PY{l+m+mi}{15}\PY{p}{,} \PY{n}{include\PYZus{}bias}\PY{o}{=}\PY{k+kc}{False}\PY{p}{)}
         \PY{n}{X\PYZus{}poly} \PY{o}{=} \PY{n}{poly\PYZus{}features}\PY{o}{.}\PY{n}{fit\PYZus{}transform}\PY{p}{(}\PY{n}{X}\PY{p}{)}
         
         \PY{c+c1}{\PYZsh{} Create Ridge regression object from Ridge class}
         \PY{n}{ridge\PYZus{}reg} \PY{o}{=} \PY{n}{Ridge}\PY{p}{(}\PY{n}{alpha}\PY{o}{=}\PY{l+m+mf}{5e\PYZhy{}2}\PY{p}{)}
         \PY{c+c1}{\PYZsh{} Fit data using Ridge regression}
         \PY{n}{ridge\PYZus{}reg}\PY{o}{.}\PY{n}{fit}\PY{p}{(}\PY{n}{X\PYZus{}poly}\PY{p}{,} \PY{n}{y}\PY{p}{)}
         
         \PY{c+c1}{\PYZsh{} Create Linear regression object from LinearRegress class (this is just for comparison)}
         \PY{n}{lin\PYZus{}reg} \PY{o}{=} \PY{n}{LinearRegression}\PY{p}{(}\PY{p}{)}
         \PY{c+c1}{\PYZsh{} Fit data using Linear regression}
         \PY{n}{lin\PYZus{}reg}\PY{o}{.}\PY{n}{fit}\PY{p}{(}\PY{n}{X\PYZus{}poly}\PY{p}{,} \PY{n}{y}\PY{p}{)}
         
         \PY{n}{y\PYZus{}predict\PYZus{}ridge} \PY{o}{=} \PY{n}{ridge\PYZus{}reg}\PY{o}{.}\PY{n}{predict}\PY{p}{(}\PY{n}{X\PYZus{}poly}\PY{p}{)}
         \PY{n}{y\PYZus{}predict\PYZus{}linear} \PY{o}{=} \PY{n}{lin\PYZus{}reg}\PY{o}{.}\PY{n}{predict}\PY{p}{(}\PY{n}{X\PYZus{}poly}\PY{p}{)}
         
         \PY{n}{plt}\PY{o}{.}\PY{n}{scatter}\PY{p}{(}\PY{n}{X}\PY{p}{,} \PY{n}{y}\PY{p}{,} \PY{n}{label}\PY{o}{=}\PY{l+s+s1}{\PYZsq{}}\PY{l+s+s1}{data}\PY{l+s+s1}{\PYZsq{}}\PY{p}{)}
         \PY{n}{plt}\PY{o}{.}\PY{n}{plot}\PY{p}{(}\PY{n}{X}\PY{p}{,} \PY{n}{y\PYZus{}predict\PYZus{}linear}\PY{p}{,} \PY{n}{color}\PY{o}{=}\PY{l+s+s1}{\PYZsq{}}\PY{l+s+s1}{b}\PY{l+s+s1}{\PYZsq{}}\PY{p}{,} \PY{n}{label}\PY{o}{=}\PY{l+s+s1}{\PYZsq{}}\PY{l+s+s1}{Linear regression fit}\PY{l+s+s1}{\PYZsq{}}\PY{p}{)}
         \PY{n}{plt}\PY{o}{.}\PY{n}{plot}\PY{p}{(}\PY{n}{X}\PY{p}{,} \PY{n}{y\PYZus{}predict\PYZus{}ridge}\PY{p}{,} \PY{n}{color}\PY{o}{=}\PY{l+s+s1}{\PYZsq{}}\PY{l+s+s1}{r}\PY{l+s+s1}{\PYZsq{}}\PY{p}{,} \PY{n}{label}\PY{o}{=}\PY{l+s+s1}{\PYZsq{}}\PY{l+s+s1}{Ridge regression fit}\PY{l+s+s1}{\PYZsq{}}\PY{p}{)}
         
         \PY{n}{plt}\PY{o}{.}\PY{n}{xlabel}\PY{p}{(}\PY{l+s+s1}{\PYZsq{}}\PY{l+s+s1}{x}\PY{l+s+s1}{\PYZsq{}}\PY{p}{)}
         \PY{n}{plt}\PY{o}{.}\PY{n}{ylabel}\PY{p}{(}\PY{l+s+s1}{\PYZsq{}}\PY{l+s+s1}{y}\PY{l+s+s1}{\PYZsq{}}\PY{p}{)}
         
         \PY{n}{plt}\PY{o}{.}\PY{n}{legend}\PY{p}{(}\PY{n}{loc}\PY{o}{=}\PY{l+s+s1}{\PYZsq{}}\PY{l+s+s1}{best}\PY{l+s+s1}{\PYZsq{}}\PY{p}{)}
\end{Verbatim}


    \begin{Verbatim}[commandchars=\\\{\}]
C:\textbackslash{}Users\textbackslash{}Sindr\textbackslash{}Anaconda3\textbackslash{}lib\textbackslash{}site-packages\textbackslash{}sklearn\textbackslash{}linear\_model\textbackslash{}ridge.py:112: LinAlgWarning: scipy.linalg.solve
Ill-conditioned matrix detected. Result is not guaranteed to be accurate.
Reciprocal condition number9.211483e-17
  overwrite\_a=True).T

    \end{Verbatim}

\begin{Verbatim}[commandchars=\\\{\}]
{\color{outcolor}Out[{\color{outcolor}33}]:} <matplotlib.legend.Legend at 0x26955f7d898>
\end{Verbatim}
            
    \begin{center}
    \adjustimage{max size={0.9\linewidth}{0.9\paperheight}}{output_55_2.png}
    \end{center}
    { \hspace*{\fill} \\}
    
    \subsection{Exercise 2.2}\label{exercise-2.2}

    \subparagraph{TODO}\label{todo}

\begin{quote}
Observing that when changing(/decreasing) the \textbf{regularization
parameter} the model starts overfitting
\end{quote}

    \section{Linear Regression with Radial Basis
Functions}\label{linear-regression-with-radial-basis-functions}

    \begin{Verbatim}[commandchars=\\\{\}]
{\color{incolor}In [{\color{incolor}34}]:} \PY{c+c1}{\PYZsh{} Set random seed}
         \PY{n}{np}\PY{o}{.}\PY{n}{random}\PY{o}{.}\PY{n}{seed}\PY{p}{(}\PY{l+m+mi}{0}\PY{p}{)}
         \PY{n}{m} \PY{o}{=} \PY{l+m+mi}{100}
         
         \PY{c+c1}{\PYZsh{} Create random set of m x values between \PYZhy{}6 and +6}
         \PY{n}{X} \PY{o}{=} \PY{n}{np}\PY{o}{.}\PY{n}{random}\PY{o}{.}\PY{n}{rand}\PY{p}{(}\PY{n}{m}\PY{p}{,} \PY{l+m+mi}{1}\PY{p}{)}\PY{o}{*}\PY{l+m+mi}{12} \PY{o}{\PYZhy{}} \PY{l+m+mi}{6}
         \PY{n}{X}\PY{o}{.}\PY{n}{sort}\PY{p}{(}\PY{n}{axis} \PY{o}{=}\PY{l+m+mi}{0}\PY{p}{)}
         
         \PY{c+c1}{\PYZsh{} Create a non\PYZhy{}linear dataset with random noise}
         \PY{n}{y} \PY{o}{=} \PY{l+m+mf}{0.5}\PY{o}{*}\PY{n}{np}\PY{o}{.}\PY{n}{cos}\PY{p}{(}\PY{n}{X}\PY{p}{)} \PY{o}{+} \PY{n}{np}\PY{o}{.}\PY{n}{sin}\PY{p}{(}\PY{n}{X}\PY{p}{)} \PY{o}{+} \PY{l+m+mi}{4}\PY{o}{*}\PY{n}{np}\PY{o}{.}\PY{n}{cos}\PY{p}{(}\PY{l+m+mi}{2}\PY{o}{*}\PY{n}{X}\PY{p}{)} \PY{o}{+} \PY{n}{np}\PY{o}{.}\PY{n}{exp}\PY{p}{(}\PY{n}{np}\PY{o}{.}\PY{n}{cos}\PY{p}{(}\PY{l+m+mi}{3}\PY{o}{*}\PY{n}{X}\PY{p}{)}\PY{p}{)} \PY{o}{+} \PY{l+m+mi}{3}\PY{o}{*}\PY{n}{np}\PY{o}{.}\PY{n}{random}\PY{o}{.}\PY{n}{rand}\PY{p}{(}\PY{n}{m}\PY{p}{,}\PY{l+m+mi}{1}\PY{p}{)} 
         
         \PY{c+c1}{\PYZsh{} plot it}
         \PY{n}{plt}\PY{o}{.}\PY{n}{scatter}\PY{p}{(}\PY{n}{X}\PY{p}{,} \PY{n}{y}\PY{p}{)}
\end{Verbatim}


\begin{Verbatim}[commandchars=\\\{\}]
{\color{outcolor}Out[{\color{outcolor}34}]:} <matplotlib.collections.PathCollection at 0x26955fe5fd0>
\end{Verbatim}
            
    \begin{center}
    \adjustimage{max size={0.9\linewidth}{0.9\paperheight}}{output_59_1.png}
    \end{center}
    { \hspace*{\fill} \\}
    
    \begin{Verbatim}[commandchars=\\\{\}]
{\color{incolor}In [{\color{incolor}35}]:} \PY{k+kn}{from} \PY{n+nn}{sklearn}\PY{n+nn}{.}\PY{n+nn}{metrics}\PY{n+nn}{.}\PY{n+nn}{pairwise} \PY{k}{import} \PY{n}{rbf\PYZus{}kernel} 
         \PY{k+kn}{from} \PY{n+nn}{sklearn}\PY{n+nn}{.}\PY{n+nn}{linear\PYZus{}model} \PY{k}{import} \PY{n}{LinearRegression}
         
         \PY{c+c1}{\PYZsh{} find the transformation of X using Radial Basis Functions }
         \PY{c+c1}{\PYZsh{} Each point in X is now modeled as vector of 100 values. }
         \PY{c+c1}{\PYZsh{} See the X\PYZus{}RBF.shape and X\PYZus{}RBF to find how rbf\PYZus{}kernel transformed }
         \PY{c+c1}{\PYZsh{} the original datapoints}
         \PY{n}{X\PYZus{}RBF} \PY{o}{=} \PY{n}{rbf\PYZus{}kernel}\PY{p}{(}\PY{n}{X}\PY{p}{,} \PY{n}{X}\PY{p}{,} \PY{n}{gamma}\PY{o}{=}\PY{l+m+mf}{0.1}\PY{p}{)} 
         
         \PY{c+c1}{\PYZsh{} Fit a linear regression model to the RBF\PYZhy{}transformed data}
         \PY{n}{clf} \PY{o}{=} \PY{n}{LinearRegression}\PY{p}{(}\PY{p}{)}
         \PY{n}{clf}\PY{o}{.}\PY{n}{fit}\PY{p}{(}\PY{n}{X\PYZus{}RBF}\PY{p}{,} \PY{n}{y}\PY{p}{)}
         
         \PY{c+c1}{\PYZsh{} find the predicted values}
         \PY{n}{y\PYZus{}}\PY{o}{=} \PY{n}{clf}\PY{o}{.}\PY{n}{predict}\PY{p}{(}\PY{n}{X\PYZus{}RBF}\PY{p}{)}
         
         \PY{c+c1}{\PYZsh{} plot original data and predicted fit}
         \PY{n}{plt}\PY{o}{.}\PY{n}{scatter}\PY{p}{(}\PY{n}{X}\PY{p}{,} \PY{n}{y}\PY{p}{,} \PY{n}{label}\PY{o}{=}\PY{l+s+s1}{\PYZsq{}}\PY{l+s+s1}{Original data}\PY{l+s+s1}{\PYZsq{}}\PY{p}{)}
         \PY{n}{plt}\PY{o}{.}\PY{n}{plot}\PY{p}{(}\PY{n}{X}\PY{p}{,} \PY{n}{y\PYZus{}}\PY{p}{,} \PY{n}{color}\PY{o}{=}\PY{l+s+s1}{\PYZsq{}}\PY{l+s+s1}{r}\PY{l+s+s1}{\PYZsq{}}\PY{p}{,} \PY{n}{label}\PY{o}{=}\PY{l+s+s1}{\PYZsq{}}\PY{l+s+s1}{Fit for RBF\PYZhy{}transformed data}\PY{l+s+s1}{\PYZsq{}}\PY{p}{)}
         \PY{n}{plt}\PY{o}{.}\PY{n}{legend}\PY{p}{(}\PY{n}{loc}\PY{o}{=}\PY{l+s+s1}{\PYZsq{}}\PY{l+s+s1}{best}\PY{l+s+s1}{\PYZsq{}}\PY{p}{)}
\end{Verbatim}


\begin{Verbatim}[commandchars=\\\{\}]
{\color{outcolor}Out[{\color{outcolor}35}]:} <matplotlib.legend.Legend at 0x26958529048>
\end{Verbatim}
            
    \begin{center}
    \adjustimage{max size={0.9\linewidth}{0.9\paperheight}}{output_60_1.png}
    \end{center}
    { \hspace*{\fill} \\}
    
    \subsection{Exercise 2.3}\label{exercise-2.3}

    \begin{Verbatim}[commandchars=\\\{\}]
{\color{incolor}In [{\color{incolor}36}]:} \PY{c+c1}{\PYZsh{} TODO}
         \PY{c+c1}{\PYZsh{} Paste your solution here}
         \PY{n}{X\PYZus{}RBF\PYZus{}0th} \PY{o}{=} \PY{n}{X\PYZus{}RBF}\PY{p}{[}\PY{l+m+mi}{0}\PY{p}{]}
         \PY{n}{X\PYZus{}RBF\PYZus{}49th} \PY{o}{=} \PY{n}{X\PYZus{}RBF}\PY{p}{[}\PY{l+m+mi}{49}\PY{p}{]}
         \PY{n}{X\PYZus{}RBF\PYZus{}99th} \PY{o}{=} \PY{n}{X\PYZus{}RBF}\PY{p}{[}\PY{l+m+mi}{99}\PY{p}{]}
         \PY{n}{ans} \PY{o}{=} \PY{n}{np}\PY{o}{.}\PY{n}{array}\PY{p}{(}\PY{p}{[}\PY{n}{X\PYZus{}RBF\PYZus{}0th}\PY{p}{,} \PY{n}{X\PYZus{}RBF\PYZus{}49th}\PY{p}{,} \PY{n}{X\PYZus{}RBF\PYZus{}99th}\PY{p}{]}\PY{p}{)}
         
         \PY{n}{plt}\PY{o}{.}\PY{n}{plot}\PY{p}{(}\PY{n}{X}\PY{p}{,} \PY{n}{ans}\PY{o}{.}\PY{n}{T}\PY{p}{)}
         \PY{n}{plt}\PY{o}{.}\PY{n}{show}\PY{p}{(}\PY{p}{)}
\end{Verbatim}


    \begin{center}
    \adjustimage{max size={0.9\linewidth}{0.9\paperheight}}{output_62_0.png}
    \end{center}
    { \hspace*{\fill} \\}
    
    \section{3. Logistic Regression }\label{logistic-regression}

    \begin{Verbatim}[commandchars=\\\{\}]
{\color{incolor}In [{\color{incolor}37}]:} \PY{k+kn}{from} \PY{n+nn}{sklearn} \PY{k}{import} \PY{n}{datasets}
         \PY{n}{iris} \PY{o}{=} \PY{n}{datasets}\PY{o}{.}\PY{n}{load\PYZus{}iris}\PY{p}{(}\PY{p}{)}
         
         \PY{c+c1}{\PYZsh{} iris is a dictionary of key\PYZhy{}value pairs. Each key\PYZhy{}value pairs contains some information about the dataset.}
         \PY{c+c1}{\PYZsh{} Lets display a list of these keys and see what they hold}
         \PY{n+nb}{list}\PY{p}{(}\PY{n}{iris}\PY{o}{.}\PY{n}{keys}\PY{p}{(}\PY{p}{)}\PY{p}{)}
\end{Verbatim}


\begin{Verbatim}[commandchars=\\\{\}]
{\color{outcolor}Out[{\color{outcolor}37}]:} ['data', 'target', 'target\_names', 'DESCR', 'feature\_names']
\end{Verbatim}
            
    \begin{Verbatim}[commandchars=\\\{\}]
{\color{incolor}In [{\color{incolor}38}]:} \PY{c+c1}{\PYZsh{} let us get the petal width. It is present in the 4th column of data}
         \PY{n}{X} \PY{o}{=} \PY{n}{iris}\PY{p}{[}\PY{l+s+s2}{\PYZdq{}}\PY{l+s+s2}{data}\PY{l+s+s2}{\PYZdq{}}\PY{p}{]}\PY{p}{[}\PY{p}{:}\PY{p}{,} \PY{l+m+mi}{3}\PY{p}{:}\PY{p}{]} 
         \PY{n}{X}\PY{o}{.}\PY{n}{sort}\PY{p}{(}\PY{n}{axis}\PY{o}{=}\PY{l+m+mi}{0}\PY{p}{)}
         
         \PY{c+c1}{\PYZsh{} lets define a binaray variable that encodes whether a flower is Iris\PYZhy{}Virginca or not}
         \PY{c+c1}{\PYZsh{} Iris\PYZus{}virginca flower is encoded as a 2 in target  }
         \PY{n}{y} \PY{o}{=} \PY{p}{(}\PY{n}{iris}\PY{p}{[}\PY{l+s+s2}{\PYZdq{}}\PY{l+s+s2}{target}\PY{l+s+s2}{\PYZdq{}}\PY{p}{]} \PY{o}{==} \PY{l+m+mi}{2}\PY{p}{)}\PY{o}{.}\PY{n}{astype}\PY{p}{(}\PY{n}{np}\PY{o}{.}\PY{n}{int}\PY{p}{)} \PY{c+c1}{\PYZsh{} 1 if Iris\PYZhy{}Virginica, else 0}
\end{Verbatim}


    \begin{Verbatim}[commandchars=\\\{\}]
{\color{incolor}In [{\color{incolor}39}]:} \PY{n}{plt}\PY{o}{.}\PY{n}{scatter}\PY{p}{(}\PY{n}{X}\PY{p}{,} \PY{n}{y}\PY{p}{)}
         \PY{n}{plt}\PY{o}{.}\PY{n}{xlabel}\PY{p}{(}\PY{l+s+s1}{\PYZsq{}}\PY{l+s+s1}{Petal width (cm)}\PY{l+s+s1}{\PYZsq{}}\PY{p}{)}
         \PY{n}{plt}\PY{o}{.}\PY{n}{ylabel}\PY{p}{(}\PY{l+s+s1}{\PYZsq{}}\PY{l+s+s1}{Iris\PYZhy{}Virginica(1) }\PY{l+s+se}{\PYZbs{}n}\PY{l+s+s1}{ Not Iris\PYZhy{}Virginica(0)}\PY{l+s+s1}{\PYZsq{}}\PY{p}{)}
\end{Verbatim}


\begin{Verbatim}[commandchars=\\\{\}]
{\color{outcolor}Out[{\color{outcolor}39}]:} Text(0,0.5,'Iris-Virginica(1) \textbackslash{}n Not Iris-Virginica(0)')
\end{Verbatim}
            
    \begin{center}
    \adjustimage{max size={0.9\linewidth}{0.9\paperheight}}{output_66_1.png}
    \end{center}
    { \hspace*{\fill} \\}
    
    \begin{Verbatim}[commandchars=\\\{\}]
{\color{incolor}In [{\color{incolor}40}]:} \PY{k+kn}{from} \PY{n+nn}{sklearn}\PY{n+nn}{.}\PY{n+nn}{linear\PYZus{}model} \PY{k}{import} \PY{n}{LinearRegression}
         
         \PY{n}{lin\PYZus{}reg} \PY{o}{=} \PY{n}{LinearRegression}\PY{p}{(}\PY{p}{)}
         \PY{n}{lin\PYZus{}reg}\PY{o}{.}\PY{n}{fit}\PY{p}{(}\PY{n}{X}\PY{p}{,} \PY{n}{y}\PY{p}{)}
         \PY{n}{y\PYZus{}} \PY{o}{=} \PY{n}{lin\PYZus{}reg}\PY{o}{.}\PY{n}{predict}\PY{p}{(}\PY{n}{X}\PY{p}{)}
         
         \PY{n}{plt}\PY{o}{.}\PY{n}{scatter}\PY{p}{(}\PY{n}{X}\PY{p}{,} \PY{n}{y}\PY{p}{,} \PY{n}{label}\PY{o}{=}\PY{l+s+s1}{\PYZsq{}}\PY{l+s+s1}{original data}\PY{l+s+s1}{\PYZsq{}}\PY{p}{)}
         \PY{n}{plt}\PY{o}{.}\PY{n}{plot}\PY{p}{(}\PY{n}{X}\PY{p}{,} \PY{n}{y\PYZus{}}\PY{p}{,} \PY{n}{color}\PY{o}{=}\PY{l+s+s1}{\PYZsq{}}\PY{l+s+s1}{r}\PY{l+s+s1}{\PYZsq{}}\PY{p}{,} \PY{n}{label}\PY{o}{=}\PY{l+s+s1}{\PYZsq{}}\PY{l+s+s1}{fit from linear regression}\PY{l+s+s1}{\PYZsq{}}\PY{p}{)}
         \PY{n}{plt}\PY{o}{.}\PY{n}{legend}\PY{p}{(}\PY{n}{loc}\PY{o}{=}\PY{l+s+s1}{\PYZsq{}}\PY{l+s+s1}{best}\PY{l+s+s1}{\PYZsq{}}\PY{p}{)}
\end{Verbatim}


\begin{Verbatim}[commandchars=\\\{\}]
{\color{outcolor}Out[{\color{outcolor}40}]:} <matplotlib.legend.Legend at 0x26958593c18>
\end{Verbatim}
            
    \begin{center}
    \adjustimage{max size={0.9\linewidth}{0.9\paperheight}}{output_67_1.png}
    \end{center}
    { \hspace*{\fill} \\}
    
    \begin{Verbatim}[commandchars=\\\{\}]
{\color{incolor}In [{\color{incolor}41}]:} \PY{k+kn}{from} \PY{n+nn}{sklearn}\PY{n+nn}{.}\PY{n+nn}{linear\PYZus{}model} \PY{k}{import} \PY{n}{LogisticRegression}
         \PY{n}{log\PYZus{}reg} \PY{o}{=} \PY{n}{LogisticRegression}\PY{p}{(}\PY{p}{)}
         \PY{n}{log\PYZus{}reg}\PY{o}{.}\PY{n}{fit}\PY{p}{(}\PY{n}{X}\PY{p}{,} \PY{n}{y}\PY{p}{)}
\end{Verbatim}


\begin{Verbatim}[commandchars=\\\{\}]
{\color{outcolor}Out[{\color{outcolor}41}]:} LogisticRegression(C=1.0, class\_weight=None, dual=False, fit\_intercept=True,
                   intercept\_scaling=1, max\_iter=100, multi\_class='ovr', n\_jobs=1,
                   penalty='l2', random\_state=None, solver='liblinear', tol=0.0001,
                   verbose=0, warm\_start=False)
\end{Verbatim}
            
    \begin{Verbatim}[commandchars=\\\{\}]
{\color{incolor}In [{\color{incolor}42}]:} \PY{c+c1}{\PYZsh{} we generate X\PYZus{}new which is vector of closely spaced points form 0 to 3}
         \PY{c+c1}{\PYZsh{} This vector will help us plot the model}
         \PY{n}{X\PYZus{}new} \PY{o}{=} \PY{n}{np}\PY{o}{.}\PY{n}{linspace}\PY{p}{(}\PY{l+m+mi}{0}\PY{p}{,} \PY{l+m+mi}{3}\PY{p}{,} \PY{l+m+mi}{1000}\PY{p}{)}\PY{o}{.}\PY{n}{reshape}\PY{p}{(}\PY{o}{\PYZhy{}}\PY{l+m+mi}{1}\PY{p}{,} \PY{l+m+mi}{1}\PY{p}{)}
         
         \PY{c+c1}{\PYZsh{} make a vector of prediction probablity values for all datapoints in X\PYZus{}new}
         \PY{n}{y\PYZus{}proba} \PY{o}{=} \PY{n}{log\PYZus{}reg}\PY{o}{.}\PY{n}{predict\PYZus{}proba}\PY{p}{(}\PY{n}{X\PYZus{}new}\PY{p}{)}
         
         \PY{n}{plt}\PY{o}{.}\PY{n}{plot}\PY{p}{(}\PY{n}{X\PYZus{}new}\PY{p}{,} \PY{n}{y\PYZus{}proba}\PY{p}{[}\PY{p}{:}\PY{p}{,} \PY{l+m+mi}{1}\PY{p}{]}\PY{p}{,} \PY{l+s+s2}{\PYZdq{}}\PY{l+s+s2}{g\PYZhy{}}\PY{l+s+s2}{\PYZdq{}}\PY{p}{,} \PY{n}{label}\PY{o}{=}\PY{l+s+s2}{\PYZdq{}}\PY{l+s+s2}{Iris\PYZhy{}Virginica Prob}\PY{l+s+s2}{\PYZdq{}}\PY{p}{)}
         \PY{n}{plt}\PY{o}{.}\PY{n}{plot}\PY{p}{(}\PY{n}{X\PYZus{}new}\PY{p}{,} \PY{n}{y\PYZus{}proba}\PY{p}{[}\PY{p}{:}\PY{p}{,} \PY{l+m+mi}{0}\PY{p}{]}\PY{p}{,} \PY{l+s+s2}{\PYZdq{}}\PY{l+s+s2}{b\PYZhy{}\PYZhy{}}\PY{l+s+s2}{\PYZdq{}}\PY{p}{,} \PY{n}{label}\PY{o}{=}\PY{l+s+s2}{\PYZdq{}}\PY{l+s+s2}{Not Iris\PYZhy{}Virginica Prob}\PY{l+s+s2}{\PYZdq{}}\PY{p}{)}
         \PY{n}{plt}\PY{o}{.}\PY{n}{scatter}\PY{p}{(}\PY{n}{X}\PY{p}{,} \PY{n}{y}\PY{p}{,} \PY{n}{label}\PY{o}{=}\PY{l+s+s1}{\PYZsq{}}\PY{l+s+s1}{data}\PY{l+s+s1}{\PYZsq{}}\PY{p}{)}
         
         \PY{n}{plt}\PY{o}{.}\PY{n}{xlabel}\PY{p}{(}\PY{l+s+s1}{\PYZsq{}}\PY{l+s+s1}{Petal width (cm)}\PY{l+s+s1}{\PYZsq{}}\PY{p}{)}
         \PY{n}{plt}\PY{o}{.}\PY{n}{ylabel}\PY{p}{(}\PY{l+s+s1}{\PYZsq{}}\PY{l+s+s1}{Probability}\PY{l+s+s1}{\PYZsq{}}\PY{p}{)}
         \PY{n}{plt}\PY{o}{.}\PY{n}{legend}\PY{p}{(}\PY{n}{loc}\PY{o}{=}\PY{l+s+s1}{\PYZsq{}}\PY{l+s+s1}{best}\PY{l+s+s1}{\PYZsq{}}\PY{p}{)}
\end{Verbatim}


\begin{Verbatim}[commandchars=\\\{\}]
{\color{outcolor}Out[{\color{outcolor}42}]:} <matplotlib.legend.Legend at 0x26958614e48>
\end{Verbatim}
            
    \begin{center}
    \adjustimage{max size={0.9\linewidth}{0.9\paperheight}}{output_69_1.png}
    \end{center}
    { \hspace*{\fill} \\}
    
    \begin{Verbatim}[commandchars=\\\{\}]
{\color{incolor}In [{\color{incolor}43}]:} \PY{n}{log\PYZus{}reg}\PY{o}{.}\PY{n}{predict}\PY{p}{(}\PY{p}{[}\PY{p}{[}\PY{l+m+mf}{1.7}\PY{p}{]}\PY{p}{]}\PY{p}{)}
\end{Verbatim}


\begin{Verbatim}[commandchars=\\\{\}]
{\color{outcolor}Out[{\color{outcolor}43}]:} array([1])
\end{Verbatim}
            
    \begin{Verbatim}[commandchars=\\\{\}]
{\color{incolor}In [{\color{incolor}44}]:} \PY{n}{log\PYZus{}reg}\PY{o}{.}\PY{n}{predict\PYZus{}proba}\PY{p}{(}\PY{p}{[}\PY{p}{[}\PY{l+m+mf}{1.7}\PY{p}{]}\PY{p}{]}\PY{p}{)}
\end{Verbatim}


\begin{Verbatim}[commandchars=\\\{\}]
{\color{outcolor}Out[{\color{outcolor}44}]:} array([[0.43834057, 0.56165943]])
\end{Verbatim}
            
    \subsection{Exercise 3.1}\label{exercise-3.1}

    \begin{Verbatim}[commandchars=\\\{\}]
{\color{incolor}In [{\color{incolor}45}]:} \PY{n}{multiclass\PYZus{}logreg\PYZus{}obj} \PY{o}{=} \PY{n}{LogisticRegression}\PY{p}{(}\PY{n}{multi\PYZus{}class}\PY{o}{=}\PY{l+s+s2}{\PYZdq{}}\PY{l+s+s2}{multinomial}\PY{l+s+s2}{\PYZdq{}}\PY{p}{,} \PY{n}{solver}\PY{o}{=}\PY{l+s+s2}{\PYZdq{}}\PY{l+s+s2}{lbfgs}\PY{l+s+s2}{\PYZdq{}}\PY{p}{,} \PY{n}{C}\PY{o}{=}\PY{l+m+mi}{10}\PY{p}{)}
         \PY{c+c1}{\PYZsh{} let us get the petal width. It is present in the 4th column of data}
         \PY{n}{X} \PY{o}{=} \PY{n}{iris}\PY{p}{[}\PY{l+s+s2}{\PYZdq{}}\PY{l+s+s2}{data}\PY{l+s+s2}{\PYZdq{}}\PY{p}{]}\PY{p}{[}\PY{p}{:}\PY{p}{,} \PY{p}{[}\PY{l+m+mi}{2}\PY{p}{,}\PY{l+m+mi}{3}\PY{p}{]}\PY{p}{]} 
         \PY{n}{y} \PY{o}{=} \PY{n}{iris}\PY{p}{[}\PY{l+s+s2}{\PYZdq{}}\PY{l+s+s2}{target}\PY{l+s+s2}{\PYZdq{}}\PY{p}{]}
\end{Verbatim}


    \begin{Verbatim}[commandchars=\\\{\}]
{\color{incolor}In [{\color{incolor}46}]:} \PY{n+nb}{print}\PY{p}{(}\PY{n}{X}\PY{o}{.}\PY{n}{shape}\PY{p}{)}
         \PY{n+nb}{print}\PY{p}{(}\PY{n}{y}\PY{o}{.}\PY{n}{shape}\PY{p}{)}
\end{Verbatim}


    \begin{Verbatim}[commandchars=\\\{\}]
(150, 2)
(150,)

    \end{Verbatim}

    \begin{Verbatim}[commandchars=\\\{\}]
{\color{incolor}In [{\color{incolor}47}]:} \PY{n}{df} \PY{o}{=} \PY{n}{pd}\PY{o}{.}\PY{n}{DataFrame}\PY{p}{(}\PY{n}{X}\PY{p}{,} \PY{n}{columns}\PY{o}{=}\PY{p}{[}\PY{n}{iris}\PY{o}{.}\PY{n}{feature\PYZus{}names}\PY{p}{[}\PY{l+m+mi}{2}\PY{p}{]}\PY{p}{,} \PY{n}{iris}\PY{o}{.}\PY{n}{feature\PYZus{}names}\PY{p}{[}\PY{l+m+mi}{3}\PY{p}{]}\PY{p}{]}\PY{p}{)}
\end{Verbatim}


    \begin{Verbatim}[commandchars=\\\{\}]
{\color{incolor}In [{\color{incolor}48}]:} \PY{n+nb}{print}\PY{p}{(}\PY{n}{df}\PY{o}{.}\PY{n}{head}\PY{p}{(}\PY{p}{)}\PY{p}{)}
\end{Verbatim}


    \begin{Verbatim}[commandchars=\\\{\}]
   petal length (cm)  petal width (cm)
0                1.4               0.1
1                1.4               0.1
2                1.3               0.1
3                1.5               0.1
4                1.4               0.1

    \end{Verbatim}

    \begin{Verbatim}[commandchars=\\\{\}]
{\color{incolor}In [{\color{incolor}49}]:} \PY{n}{multiclass\PYZus{}logreg\PYZus{}obj}\PY{o}{.}\PY{n}{fit}\PY{p}{(}\PY{n}{X}\PY{p}{,} \PY{n}{y}\PY{p}{)}
\end{Verbatim}


\begin{Verbatim}[commandchars=\\\{\}]
{\color{outcolor}Out[{\color{outcolor}49}]:} LogisticRegression(C=10, class\_weight=None, dual=False, fit\_intercept=True,
                   intercept\_scaling=1, max\_iter=100, multi\_class='multinomial',
                   n\_jobs=1, penalty='l2', random\_state=None, solver='lbfgs',
                   tol=0.0001, verbose=0, warm\_start=False)
\end{Verbatim}
            
    \begin{Verbatim}[commandchars=\\\{\}]
{\color{incolor}In [{\color{incolor}69}]:} \PY{n}{X\PYZus{}new} \PY{o}{=} \PY{n}{np}\PY{o}{.}\PY{n}{array}\PY{p}{(}\PY{p}{[}\PY{l+m+mi}{1}\PY{p}{,} \PY{l+m+mf}{0.1}\PY{p}{]}\PY{p}{,} \PY{n}{ndmin}\PY{o}{=}\PY{l+m+mi}{2}\PY{p}{)}
         \PY{n}{prediction} \PY{o}{=} \PY{n}{multiclass\PYZus{}logreg\PYZus{}obj}\PY{o}{.}\PY{n}{predict}\PY{p}{(}\PY{n}{X\PYZus{}new}\PY{p}{)}
         \PY{n}{pred\PYZus{}class} \PY{o}{=} \PY{n}{prediction}\PY{p}{[}\PY{l+m+mi}{0}\PY{p}{]}
         \PY{n}{pred\PYZus{}prob} \PY{o}{=} \PY{n}{multiclass\PYZus{}logreg\PYZus{}obj}\PY{o}{.}\PY{n}{predict\PYZus{}proba}\PY{p}{(}\PY{n}{X\PYZus{}new}\PY{p}{)}
         \PY{n}{pred\PYZus{}prob\PYZus{}class} \PY{o}{=} \PY{p}{(}\PY{n}{pred\PYZus{}prob}\PY{p}{[}\PY{l+m+mi}{0}\PY{p}{]}\PY{p}{[}\PY{n}{pred\PYZus{}class}\PY{p}{]}\PY{p}{)}
         \PY{n+nb}{print}\PY{p}{(}\PY{l+s+s1}{\PYZsq{}}\PY{l+s+s1}{Prediction }\PY{l+s+si}{\PYZob{}\PYZcb{}}\PY{l+s+s1}{\PYZsq{}}\PY{o}{.}\PY{n}{format}\PY{p}{(}\PY{n}{prediction}\PY{p}{)} \PY{o}{+} \PY{l+s+s1}{\PYZsq{}}\PY{l+s+s1}{, with a probability of: }\PY{l+s+si}{\PYZob{}\PYZcb{}}\PY{l+s+s1}{\PYZsq{}}\PY{o}{.}\PY{n}{format}\PY{p}{(}\PY{n}{pred\PYZus{}prob\PYZus{}class}\PY{p}{)}\PY{p}{)}
\end{Verbatim}


    \begin{Verbatim}[commandchars=\\\{\}]
Prediction [0], with a probability of: 0.9995796959204409

    \end{Verbatim}


    % Add a bibliography block to the postdoc
    
    
    
    \end{document}
